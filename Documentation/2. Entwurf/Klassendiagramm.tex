	\newcommand{\class}[1]{\paragraph{Klasse #1:}\ \\ }
	\newcommand{\method}[1]{\textcolor{blue}{#1}}
	\newcommand{\kursiv}[1]{{\it #1}}
	\newcommand{\override}{{\it @Override}\ \\}
	
	\chapter{Klassenbeschreibung und Klassendiagramm}
	In diesem Kapitel werden alle Klassen der Anwendung \textbf{ofCourse} aufgeführt.
	Um eine bessere Übersicht und Strukturierung zu erhalten wird das ganze Projekt in Pakete aufgeteilt.
	

	\section{Package Action}
		\begin{tiny}
			TF
		\end{tiny}
	\subsection{Klasse AuthenticateUser}
	\kursiv{ManagedBean, RequestScoped}\\
	Die Klasse ist für die Authentifizierung eines Benutzers im System zuständig.
	\begin{itemize}
		\item \method{public boolean login()}
		\item \method{public boolean logout()}
		\item \method{public User getLoginUser()}
		\item \method{public void setLoginUser(User userToLogIn)}
		\item \method{public SessionUser getSessionUser()}
		\item \method{public void setSessionUser(SessionUser userSession)}
	\end{itemize}
	
	\subsection{Klasse RegisterUser}
	\kursiv{ManagedBean, RequestScoped}\\
	Die Klasse ist für die Registrierung eines Benutzers im System zuständig.
	\begin{itemize}
		\item \method{public int registerUser()}
		\item \method{public User getUsertoRegistrate()}
		\item \method{public void setUserToRegistrate(User userToRegistrate)}
	\end{itemize}
	
	\subsection{Klasse LostPassword}
	\kursiv{ManagedBean, RequestScoped}\\
	Die Klasse ist die 'Passwort  vergessen' - Funktion zuständig, d.h. sie ersetzt das alte Passwort eines Benutzers durch ein zufällig generiertes und sendet es an die eingegebene E-Mailadresse.
	\begin{itemize}
		\item \method{public boolean resetPassword()}
		\item \method{public String getEmailAddressToResetPassword()}
		\item \method{public void setEmailAddressToResetPassword(String emailToResetPassword)}
	\end{itemize}
	
	\subsection{Klasse SearchUser}
	\kursiv{ManagedBean, ViewScoped}\\
	Diese Klasse stellt den Mechanismus zum Suchen von Benutzern zur Verfügung.
	\begin{itemize}
		\item \method{public ArrayList<User> getSearchResult()}
		\item \method{public void setSearchResult(ArrayList<User> searchResult)}
		\item \method{public void search(String searchParam, String searchTerm)}
		\item \method{public String getSearchParam()}
		\item \method{public void setSearchParam(String searchParam)}
		\item \method{public String getSearchTerm()}
		\item \method{public void setSearchTerm(String searchTerm)}
		\item \method{public void sortBySpecicColumn()}
		\item \method{public int getActualPageNumber()}
		\item \method{public void goToSpecicPage()}
		\item \method{public Pagination getPagination()}
		\item \method{public void setPagination(Pagination pagination)}
		\item \method{public SessionUser getSessionUser()}
		\item \method{public void setSessionUser(SessionUser userSession)}
	\end{itemize}
	
	\subsection{Klasse SearchCourse}
	\kursiv{ManagedBean, ViewScoped}\\
	Diese Klasse stellt den Mechanismus zum Suchen von Kursen zur Verfügung. Außerdem wird darin die Einschränkung des Anzeigezeitraums realisiert.
	\begin{itemize}
		\item \method{public void searchForFreeCourses()}
		\item \method{public void displayCoursesInSpecificPeriod()}
		\item \method{public String getDisplayPeriod()}
		\item \method{public void setDisplayPeriod(String displayPeriod)}
		\item \method{public ArrayList<User> getSearchResult()}
		\item \method{public void setSearchResult(ArrayList<User> searchResult)}
		\item \method{public void search(String searchParam, String searchTerm)}
		\item \method{public String getSearchParam()}
		\item \method{public void setSearchParam(String searchParam)}
		\item \method{public String getSearchTerm()}
		\item \method{public void setSearchTerm(String searchTerm)}
		\item \method{public void sortBySpecicColumn()}
		\item \method{public int getActualPageNumber()}
		\item \method{public void goToSpecicPage()}
		\item \method{public Pagination getPagination()}
		\item \method{public void setPagination(Pagination pagination)}
		\item \method{public SessionUser getSessionUser()}
		\item \method{public void setSessionUser(SessionUser userSession)}
	\end{itemize}
	
	\subsection{Klasse UserProfil}
	\kursiv{ManagedBean, ViewScoped}\\
	Diese Klasse ist zuständig für die Anzeige und das Bearbeiten der Beutzerdaten. Des Weiteren realisiert die Darstellung der Liste aller vom Benutzer geleiteten Kurse.
	\begin{itemize}
		\item \method{public void editUserdata()}
		\item \method{public void saveUserdata()}
		\item \method{public boolean uploadProfilPic()}
		\item \method{public void setUserInactive()}
		\item \method{public void depositMoneyPerCreditcard()}
		\item \method{public User getUser()}
		\item \method{public void setUser(UseruserToSet)}
		\item \method{public ArrayList<Course> getManagedCourses()}
		\item \method{public void setManagedCourses(ArrayList<Course> managedCourses)}
		\item \method{public int getActualPageNumber()}
		\item \method{public void goToSpecicPage()}
		\item \method{public Pagination getPagination()}
		\item \method{public void setPagination(Pagination pagination)}
		\item \method{public SessionUser getSessionUser()}
		\item \method{public void setSessionUser(SessionUser userSession)}
	\end{itemize}
	
	\subsection{Klasse CourseDetail}
	\kursiv{ManagedBean, ViewScoped}\\
	\subsection{Klasse CourseEdit}
	\kursiv{ManagedBean, ViewScoped}\\
	\subsection{Klasse CourseUnit}
	\kursiv{ManagedBean, ViewScoped}\\
	\subsection{Klasse ContactUsers}
	\kursiv{ManagedBean, RequestScoped}\\
	\subsection{Klasse Imprint}
	\kursiv{ManagedBean, ViewScoped}\\
	Die Klasse ist für die Darstellung und Bearbeitung des Impressums verantwortlich.
	\begin{itemize}
		\item \method{public String getImprint()}
		\item \method{public void setImprint(String Imprint)}
	\end{itemize}
	\subsection{Klasse MyCourses}
	\kursiv{ManagedBean, RequestScoped}\\
	\subsection{Klasse CourseManagement}
	\kursiv{ManagedBean, ViewScoped}\\
	\subsection{Klasse UserManagement}
	\kursiv{ManagedBean, ViewScoped}
	\subsection{Klasse SystemConfiguration}
	\kursiv{ManagedBean, ViewScoped}\\
	\subsection{Klasse PaymentOnline}
	\kursiv{ManagedBean, ViewScoped}\\
	\subsection{Klasse PaymentOffline}
	\kursiv{ManagedBean, ViewScoped}\\
	Klasse ist zuständig für die Offline - Aufladung des Guthabenkontos von Benutzern durch den Administrator und um die Bezahlungseinstellungen zu verwalten.
	\subsection{Klasse IncomeStatistics}
	\kursiv{ManagedBean, ViewScoped}\\
	Die Klasse ist für die Generierung und Darstellung der Einnahmestatistiken zuständig.
	\begin{itemize}
	    \item \method{public void displayStatistic(String displayParam)}
		\item \method{public String getDisplayParam()}
		\item \method{public void setDisplayParam(String displayParam)}
		\item \method{public int getActualPageNumber()}
		\item \method{public void goToSpecicPage()}
		\item \method{public Pagination getPagination()}
		\item \method{public void setPagination(Pagination pagination)}
		\item \method{public SessionUser getSessionUser()}
		\item \method{public void setSessionUser(SessionUser userSession)}
	\end{itemize}
	\subsection{Klasse SetupAdmin}
	\kursiv{ManagedBean, RequestScoped}\\
	Diese Klasse ist zuständig für das Erstellen des ersten Systemadministrators.
	\subsection{Klasse Header}
	\kursiv{ManagedBean, RequestScoped}\\
	Diese Klasse ist zuständig für das Laden der Anmeldeseite, das Abmelden und der Auswahl der Anzeigesprache.
	\begin{itemize}
		\item \method{public boolean login()}
		\item \method{public boolean logout()}
		\item \method{public String getChoosenLanguage()}
		\item \method{public void setChoosenLanguage(String choosenLanguage)}
		\item \method{public SessionUser getSessionUser()}
		\item \method{public void setSessionUser(SessionUser userSession)}
	\end{itemize}
	
	\subsection{Klasse Footer}
	\kursiv{ManagedBean, RequestScoped}\\
	Diese Klasse ist zuständig für das Laden der AGB, der Hilfe und der Impressumsseite.
	\begin{itemize}
		\item \method{public void loadImprintPage()}
		\item \method{public void loadAGBPage()}
		\item \method{public void loadHelpPage()}
		\item \method{public SessionUser getSessionUser()}
		\item \method{public void setSessionUser(SessionUser userSession)}
	\end{itemize}
	
	\subsection{Klasse Scheduler}
	\kursiv{ManagedBean, RequestScoped}\\
	Diese Klasse ist zuständig für die Generierung und Darstellung des persönlichen Terminplaners des Benutzers.
	\begin{itemize}
		\item \method{public void displayScheduler(ArrayList<CourseUnit> bookedCourseUnits)}
		\item \method{public SessionUser getSessionUser()}
		\item \method{public void setSessionUser(SessionUser userSession)}
	\end{itemize}
	
	\subsection{Klasse SessionUser}
	\kursiv{ManagedBean, SessionScoped}\\
	Die Klasse speichert Informationen über die Session eines Benutzers. Gespeichert werden die ID des Benutzers, dessen Benutzerrolle, die gewählte Sprache.
	\begin{itemize}
		\item \method{public int getUserID()}
		\item \method{public void setUserID(int userID)}
		\item \method{public String getUserRole()}
		\item \method{public void setUserRole(String userRole)}
		\item \method{public String getLanguage()}
		\item \method{public void setLanguage(String language)}
	\end{itemize}
	
	\subsection{Klasse Mail}
	\kursiv{ManagedBean, ApplicationScoped}\\
	Die Klasse ist für die E-Mailbenachrichtigung der Benutzer zuständig. Sie ist sowohl für die automatisch gesendeten E-Mails, wie unter anderem die Verifizierung oder Accountaktivierungsbestätigung, als auch für die von Kursleitern gesendeten Mails zuständig.
	\begin{itemize}
		\item \method{public boolean sendVerificationMail(int UserID)}
		\item \method{public boolean sendConfirmaitionMail(int UserID)}
		\item \method{public boolean sendMail(String subject, String message, ArrayList<String> mailRecipients)}
	\end{itemize}
	
	\section{Package Model}
	
	\section{Package Services}
	
	\section{Package Interfaces}
	\kursiv{Nur Kommentar: Interfaces für DAO - Damit die Datenbank einfach ausgetauscht werden kann}
	
	\section{Package Database}
	
	\subsection{Klasse DatabaseConnectionManager}
	Die Klasse ist zuständig für das Aufbauen der Verbindungen zur Datenbank. Des Weiteren
	speichert und verwaltet sie die Datenbankverbindungen. 
	\begin{itemize}
		\item \method{public synchronized Connection getConnection()}
		\item \method{public synchronized void releaseConnection()}
		\item \method{public static DatabaseConnectionManager getInstance()}
		\item \method{public void shutDown()}
	\end{itemize}
	
	\section{Package System}
	\begin{tiny}
		TF
	\end{tiny}
	\subsection{Klasse CheckPhase}
	Die Klasse ist zuständig für die Überprüfung, ob der jeweilige Benutzer die Berechtigung besitzt, auf die angeforderte Seite zu gelangen.\\
	Die Klasse implementiert das Interface PhaseListener.
	\begin{itemize}
		\item \method{public PhaseId getPhaseId()}
		\item \override
		\method{public void beforePhase(PhaseEvent arg0)}
		\item \override
		\method{public void afterPhase(PhaseEvent event)}	
	\end{itemize}
	
	\subsection{Klasse Maintenance}
	Die Klasse ist zuständig dafür, dass Kurse sechs Monate nach ihrem Enddatum automatisch aus dem System gelöscht werden.\\
	Die Klasse implementiert das Interface Runnable.
	\begin{itemize}
		\item \method{public boolean isMaintenaceStopped()}
		\item \method{public synchronized void shutDown()}
		\item \method{public static Maintenance getInstance()}
		\item \override
		\method{public void run()}
	\end{itemize}
	
	\subsection{LaunchSystem}
	\kursiv{ManagedBean, ApplicationScoped}\\
	Die Klasse ist zuständig für das Starten des System. Sie startet den Maintenance - Thread, lädt Properties und stellt die Datenbankverbindung her.
	\begin{itemize}
		\item \kursiv{@PostConstruct}\\
		\method{public void startSystem()}
		\item \kursiv{@PreDestroy}\\
		\method{public void shutdownMaintenance()}
	\end{itemize}
	
	
	
	\section{Package Util}
	\begin{tiny}
		TF
	\end{tiny}
	\subsection{Klasse PasswordHash}
	Diese Klasse ist zuständig für das Hashen der Passwörter.
	\begin{itemize}
		\item \method{public static String hash(String password, int salt)}
	\end{itemize}
	
	\subsection{Klasse LanguageMangager}
	Diese Klasse ist zuständig für die angezeigte Systemsprache. In ihr werden die unterstützten Sprachen verwaltet und sie ist zuständig für das Auslesen der Anzeigetexte aus der .properties - Datei der gewählten Sprache.
	\begin{itemize}
		\item \method{public static LanguageManager getInstance()}
		\item \method{public LinkedHashMap<String, Object> getSupportedLanguages()}
		\item \method{public String getProperty(String key)}
		\item \method{public void switchLanguage(String language)}
	\end{itemize}
	
	\subsection{Klasse PropertyManager}
	Die Klasse ist zuständig für das Auslesen und Bearbeiten der Property - Datei, welche die Daten für die Systemkonfiguration, also die Daten für die Datenbankverbindung und den EMail - Service enthält.
	\begin{itemize}
		\item \method{public static PropertyManager getInstance()}
		\item \method{public String getProperty(String key)}
		\item \method{public String setProperty(String key)}
	\end{itemize}
	
	\subsection{Klasse IDGenerator}
	Die Klasse ist zuständig für das Generieren einer im System eindeutigen Identifikationsnummer für einen Benutzer, einen Kurs oder eine Kurseinheit.
	\begin{itemize}
		\item \method{public static int generateUserID()}
		\item \method{public static int generateCourseID()}
		\item \method{public static int generateCourseUnitID(int CourseID)}
	\end{itemize}
	
	\subsection{Klasse RandomSaltGenerator}
	Die Klasse ist zuständig für das Generieren eines zufälligen Salt, welcher für das Hashen des Benutzerpassworts benötigt wird.
	\begin{itemize}
		\item \method{public static int generateRandomSalt()}
	\end{itemize} 
	
	\subsection{Klasse EncodingFilter}
	Diese Klasse implementiert die Filter - Methoden um UTF-8 Encoding zu
	ermöglichen und somit Probleme bei der Zeichendarstellung zu verhindern.
	\begin{itemize}
		\item \override
		\method{public void destroy()}
		\item \override
		\method{public void doFilter(ServletRequest request, ServletResponse response,
			FilterChain chain)}
		\item \override
		\method{public void init(FilterConfig filterConfig)}
	\end{itemize}
	
	\section{Package Exception}
	\subsection{BankAccountException}
	Sind Fehler, welche beim Aufladen des Guthabenkontos auftreten. 
	
	\section{Package Validator}
	\begin{tiny}
		TF\\
	\end{tiny}\\
	Dieser Abschnitt beschäftigt sich mit den benötigten Validatoren, die notwendig sind, um die Eingaben des Benutzers zu überprüfen und gegebenenfalls Fehlermeldungen zu generieren.\\
	Alles Klassen dieses Packages implementieren das Interface Validator.
	
	\subsection{Klasse UserNameValidator}
	Der Validator überprüft, ob der eingegebene Benutzername schon im System vergeben ist.
	\begin{itemize}
		\item \override
		\method{public void validate()}
	\end{itemize}
	
	\subsection{Klasse EMailValidator}
	Der Validator überprüft, ob die Eingabe ein gültiges E-Mail-Format besitzt und ob die eingegebene E-Mailadresse bereits im System existiert.
	\begin{itemize}
		\item \override
		\method{public void validate()}
	\end{itemize}
	
	\subsection{Klasse PasswordValidator}
	Dieser Validator überprüft, ob das eingegebene Passwort gewisse Sicherheitsanforderungen bezüglich Länge und Zeichenwahl erfüllt. Vorgesehen Anforderungen an das Passwort sind mindestens 8 Zeichen, mindestens ein Sonderzeichen, mindestens eine Ziffer, Verwendung von Groß- und Kleinbuchstaben. Außerdem dürfen im Passwort keine Umlaute sowie kein 'ß' vorkommen.
	\begin{itemize}
		\item \override
		\method{public void validate()}
	\end{itemize}
	
	\subsection{Klasse ConfirmPasswordValidator}
	Dieser Validator überprüft zwei Passwörter auf ihre Übereinstimmung.
	\begin{itemize}
		\item \override
		\method{public void validate()}
	\end{itemize}
	
	\subsection{Klasse BirthdateValidator}
	Der Validator überprüft, ob das eingegebene Datum eingegebene Datum in der Zukunft liegt und ob es mehr als 150 Jahre zurückliegt.
	\begin{itemize}
		\item \override
		\method{public void validate()}
	\end{itemize}
	
	\subsection{Klasse UserImageValidator}
	Dieser Validator überprüft eine Bilddatei auf die richtige Dateiendung .jpg. Zusätzlich wird überprüft, ob
	die maximale Dateigröße und die maximal zugelassene Auflösung für ein Profilbild eines Benutzers eingehalten wird.
	\begin{itemize}
		\item \override
		\method{public void validate()}
	\end{itemize}
	
	\subsection{Klasse DateValidator}
	Der Validator überprüft Datumseingaben.
	\begin{itemize}
		\item \override
		\method{public void validate()}
	\end{itemize}
	
	\subsection{Klasse CreditCardValidator}
	Der Validator überprüft, ob eine Kreditkartenummer gültig ist.
	\begin{itemize}
		\item \override
		\method{public void validate()}
	\end{itemize}
	
	\subsection{Klasse CVCValidator}
	Dieser Validator überprüft eine CVC Nummer auf ihre Gültigkeit.
	\begin{itemize}
		\item \override
		\method{public void validate()}
	\end{itemize}
	
	\subsection{Klasse InputTextValidator}
	Überprüft den eingegebenen Text auf korrekte Zeichenkodierung.
	\begin{itemize}
		\item \override
		\method{public void validate()}
	\end{itemize}
	
	\subsection{Klasse OfflineTransactionValidator}
	Dieser Validator überprüft, ob bei einer Offline-Aufladung des Guthabenkontos eines Benutzers der eingegebene Name und die eingegebenen Benutzeridentifikationsnummer auch zum selben Benutzer gehören.
	\begin{itemize}
		\item \override
		\method{public void validate()}
	\end{itemize}
	
	\subsection{Klasse PriceValidator}
	Der Validator überprüft, ob der eingegebene Preis das korrekte Format hat, dass heißt, ob die Zahl nicht-negativ ist und zwei Nachkommastellen besitzt.
	\begin{itemize}
		\item \override
		\method{public void validate()}
	\end{itemize}
	
	\subsection{Klasse LogoImageValidator}
	Dieser Validator überprüft eine Bilddatei auf die richtige Dateiendung .jpg. Zusätzlich wird überprüft, ob
	die maximale Dateigröße und die maximal zugelassene Auflösung für ein Anwendungslogo eingehalten wird.
	\begin{itemize}
		\item \override
		\method{public void validate()}
	\end{itemize}
	
	\subsection{Klasse CustomStyleCSSValidator}
	Der Validator überprüft, ob die Datei den Namen 'customStyle' besitzt und ob es sich um den Dateityp mit der Endung .css handelt.
	\begin{itemize}
		\item \override
		\method{public void validate()}
	\end{itemize}
	
	
	
	
	\section{Verwendete Libraries}
	\begin{tiny}
		TF\\
	\end{tiny}\\
	In dem System verwendete Libraries:
	\begin{itemize}
		\item Commons Fileupload: Library für das Hochladen von Dateien.
		\item JavaMail: Library für das Versenden von E-Mails.
		\item JFreeChart: Library für die Erstellung von Diagrammen.
		\item Log4J: Library für das Loggen von Meldungen.
	\end{itemize}
	\section{Klassendiagramm}
