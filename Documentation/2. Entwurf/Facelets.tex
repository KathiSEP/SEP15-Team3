\chapter{Facelets}

	Dieses Kapitel enthält alle benötigten  Facelets. Jedes Facelet wird in unterschiedliche  Bereiche aufgeteilt:
	\begin{itemize}
		\item \textbf{Beschreibung} beinhaltet die Funktionen des Facelets.
		\item \textbf{Links} enthält die Seiten, auf welche weitergeleitet wird.
		\item \textbf{Buttons} listet die sich auf der Seite befindlichen Buttons mit deren hinterlegten Methoden auf.
		\item \textbf{Inputs} enthält eine Liste aller Eingabefelder auf der Seite.
		\item \textbf{Outputs} enthält eine Liste aller Ausgabefelder auf der Seite.
		\item \textbf{Backing Bean} gibt das mit dem Facelet verbundene Bean an.
	\end{itemize}
	
	\section{Templates}
		Kathi, Ricki
	
		\paragraph{layout.xhtml}
		
		\paragraph{header.xhtml}
		
		\paragraph{footer.xhtml}
	
	\section{Facelets}
	
		\subsection{open}
			
			\subsubsection{common}
			
				\paragraph{index.xhtml}
				Ricki
				
				\paragraph{authenticate.xhtml}
				KH\\
				\begin{itemize}
					\item \textbf{Beschreibung:}
					Auf dieser Seite kann man sich im System anmelden, ein neues Benutzerkonto generieren oder ein neues Passwort anfordern.
					\item \textbf{Links:} -
					\item \textbf{Buttons:}
						\begin{itemize}
							\item Registrieren: Durch Drücken dieses Buttons wird der neue Benutzer mit den eingegebenen Daten im System gespeichert und es wird eine Bestätigungsmail mit dem Verifizierungslink an die angegebene E-Mail-Adresse verschickt, sofern alle Daten korrekt eingegeben wurden.
							\item Anmelden: Durch Drücken dieses Buttons wird der registrierte Benutzer im System angemeldet und auf die Seite 'Meine Kurse' weitergeleitet, sofern Benutzername und Passwort korrekt eingegeben wurden.
							\item Neues Passwort anfordern: Durch Drücken dieses Buttons wird an die angegebene E-Mail-Adresse ein automatisch generiertes Passwort geschickt, sofern die Adresse im System existiert.
						\end{itemize}
					\item \textbf{Inputs:}
						\begin{itemize}
							\item Anrede (Registrierung): Hier wählt der Benutzer die Anrede 'Herr' oder 'Frau' aus.
							\item Vorname (Registrierung): Hier trägt der Benutzer seinen Vornamen ein.
							\item Name (Registrierung): Hier gibt der Benutzer seinen Namen ein.
							\item Benutzername (Registrierung): Hier gibt der Benutzer einen Benutzernamen ein.
							\item Passwort (Registrierung): Hier trägt der Benutzer ein Passwort ein.
							\item Passwort bestätigen (Registrierung): Hier gibt der Benutzer das gleiche Passwort erneut ein zur Bestätigung.
							\item Geburtstag (Registrierung): Hier gibt der Benutzer sein Geburtsdatum ein.
							\item Straße/Hausnummer (Registrierung): Hier gibt der Benutzer seine Straße und seine Hausnummer ein.
							\item Ort (Registrierung): Hier trägt der Benutzer seinen Ort ein.
							\item Postleitzahl (Registrierung): Hier trägt der Benutzer seine Postleitzahl ein.
							\item E-Mail-Adresse (Registrierung): Hier gibt der Benutzer seine E-Mail-Adresse ein.
							\item AGBs bestätigen (Registrierung): Durch Setzten des Häkchens bestätigt der Benutzer die AGBs. 
							\item Benutzername (Anmeldung): Der Benutzer gibt seinen Benutzernamen ein, mit dem er sich registriert hat.
							\item Passwort (Anmeldung): Der Benutzer gibt sein Passwort ein, mit dem er sich registriert hat.
							\item E-Mail-Adresse (Passwort vergessen): Der Benutzer gibt seine im System bereits gespeicherte E-Mailadresse ein.
						\end{itemize}
					\item \textbf{Outputs:} 
						\begin{itemize}
							\item Vorname Fehlermeldung (Registrierung): Ausgabe der Fehlermeldungen zu den Validatoren des Eingabefeldes.
							\item Name Fehlermeldung (Registrierung): Ausgabe der Fehlermeldungen zu den Validatoren des Eingabefeldes.
							\item Benutzername Fehlermeldung (Registrierung): Ausgabe der Fehlermeldungen zu den Validatoren des Eingabefeldes.
							\item Passwort Fehlermeldung (Registrierung): Ausgabe der Fehlermeldungen zu den Validatoren des Eingabefeldes.
							\item Passwort bestätigen Fehlermeldung (Registrierung): Ausgabe der Fehlermeldungen zu den Validatoren des Eingabefeldes.
							\item Geburtstag Fehlermeldung (Registrierung): Ausgabe der Fehlermeldungen zu den Validatoren des Eingabefeldes.
							\item Straße/Hausnummer Fehlermeldung (Registrierung): Ausgabe der Fehlermeldungen zu den Validatoren des Eingabefeldes.
							\item Ort Fehlermeldung (Registrierung): Ausgabe der Fehlermeldungen zu den Validatoren des Eingabefeldes.
							\item Postleitzahl Fehlermeldung (Registrierung): Ausgabe der Fehlermeldungen zu den Validatoren des Eingabefeldes.
							\item E-Mail-Adresse Fehlermeldung (Registrierung): Ausgabe der Fehlermeldungen zu den Validatoren des Eingabefeldes.
							\item AGBs bestätigen Fehlermeldung (Registrierung): Ausgabe der Fehlermeldungen zu den Validatoren des Eingabefeldes.
							\item Benutzername Fehlermeldung (Anmeldung): Ausgabe der Fehlermeldungen zu den Validatoren des Eingabefeldes.
							\item Passwort Fehlermeldung (Anmeldung): Ausgabe der Fehlermeldungen zu den Validatoren des Eingabefeldes.
							\item E-Mail-Adresse Fehlermeldung (Passwort vergessen): Ausgabe der Fehlermeldungen zu den Validatoren des Eingabefeldes.
						\end{itemize}
					\item \textbf{BackingBean:} authenticate.java
				\end{itemize}
				
				\paragraph{imprint.xhtml}
					KH\\
					\begin{itemize}
						\item \textbf{Beschreibung:}
						\item \textbf{Links:}
						\item \textbf{Buttons:}
						\item \textbf{Inputs:}
						\item \textbf{Outputs:}
						\item \textbf{BackingBean:}
					\end{itemize}
				
				
				\paragraph{help.xhtml}
				Ricki
				
				\paragraph{agb.xhtml}
				Ricki
		
			\subsubsection{courses}
				
				\paragraph{listCourses.xhtml}
				Ricki
				
				\paragraph{courseDetails.xhtml}
				Ricki
		
		\subsection{users}
		
			\subsubsection{registeredUser}
				
				\paragraph{myCourses.xhtml}
					KH\\
					\begin{itemize}
						\item \textbf{Beschreibung:}
						\item \textbf{Links:}
						\item \textbf{Buttons:}
						\item \textbf{Inputs:}
						\item \textbf{Outputs:}
						\item \textbf{BackingBean:}
					\end{itemize}
				
				\paragraph{profile.xhtml}
					KH\\
					\begin{itemize}
						\item \textbf{Beschreibung:}
						\item \textbf{Links:}
						\item \textbf{Buttons:}
						\item \textbf{Inputs:}
						\item \textbf{Outputs:}
						\item \textbf{BackingBean:}
					\end{itemize}
				
				\paragraph{buyCredits.xhtml}
				Ricki
				
				\paragraph{scheduler.xhtml}
				Ricki
				
				\paragraph{leaderProfile.xhtml}
					KH\\
					\begin{itemize}
						\item \textbf{Beschreibung:}
						\item \textbf{Links:}
						\item \textbf{Buttons:}
						\item \textbf{Inputs:}
						\item \textbf{Outputs:}
						\item \textbf{BackingBean:}
					\end{itemize}
				
				\paragraph{listParticipants.xhtml}
					KH\\
					\begin{itemize}
						\item \textbf{Beschreibung:}
						\item \textbf{Links:}
						\item \textbf{Buttons:}
						\item \textbf{Inputs:}
						\item \textbf{Outputs:}
						\item \textbf{BackingBean:}
					\end{itemize}
			
			\subsubsection{courseLeader}
			
				\paragraph{editCourseUnit.xhtml}
					KH\\
					\begin{itemize}
						\item \textbf{Beschreibung:}
						\item \textbf{Links:}
						\item \textbf{Buttons:}
						\item \textbf{Inputs:}
						\item \textbf{Outputs:}
						\item \textbf{BackingBean:}
					\end{itemize}
				
				\paragraph{listUsers.xhtml}
					KH\\
					\begin{itemize}
						\item \textbf{Beschreibung:}
						\item \textbf{Links:}
						\item \textbf{Buttons:}
						\item \textbf{Inputs:}
						\item \textbf{Outputs:}
						\item \textbf{BackingBean:}
					\end{itemize}
			
			\subsubsection{systemAdministrator}
			
				\paragraph{adminManagement.xhtml}
				Ricki
				
				\paragraph{createUser.xhtml}
				Ricki
				
				\paragraph{createCourse.xhtml}
					KH\\
					\begin{itemize}
						\item \textbf{Beschreibung:}
						\item \textbf{Links:}
						\item \textbf{Buttons:}
						\item \textbf{Inputs:}
						\item \textbf{Outputs:}
						\item \textbf{BackingBean:}
					\end{itemize}
				
				\paragraph{editImprint.xhtml}
					KH\\
					\begin{itemize}
						\item \textbf{Beschreibung:}
						\item \textbf{Links:}
						\item \textbf{Buttons:}
						\item \textbf{Inputs:}
						\item \textbf{Outputs:}
						\item \textbf{BackingBean:}
					\end{itemize}
				
				\paragraph{statistics.xhtml}
				Ricki