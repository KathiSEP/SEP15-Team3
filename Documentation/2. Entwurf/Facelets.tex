\chapter{Facelets}

	Dieses Kapitel enthält alle benötigten  Facelets. Jedes Facelet wird in unterschiedliche  Bereiche aufgeteilt:
	\begin{itemize}
		\item \textbf{Beschreibung} beinhaltet die Funktionen des Facelets.
		\item \textbf{Links} enthält die Seiten, auf welche weitergeleitet wird.
		\item \textbf{Buttons} listet die sich auf der Seite befindlichen Buttons mit deren hinterlegten Methoden auf.
		\item \textbf{Inputs} enthält eine Liste aller Eingabefelder auf der Seite.
		\item \textbf{Outputs} enthält eine Liste aller Ausgabefelder auf der Seite.
		\item \textbf{Backing Bean} gibt das mit dem Facelet verbundene Bean an.
	\end{itemize}
	
	\section{Templates}
		Kathi, Ricky
	
		\paragraph{layout.xhtml}
		
		\paragraph{header.xhtml}
		
		\paragraph{footer.xhtml}
	
	\section{Facelets}
	
		\subsection{open}
			
			\subsubsection{common}
			
				\paragraph{index.xhtml}
					RS\\
					\begin{itemize}
						\item \textbf{Beschreibung:} Dieses Facelet stellt die Startseite des Systems dar.
						\item \textbf{Links:}
							\begin{itemize}
								\item Gesamtes Kursangebot, listCourses.xhtml
							\end{itemize}
						\item \textbf{Buttons:} -
						\item \textbf{Inputs:} -
						\item \textbf{Outputs:}
							\begin{itemize}
								\item Logo der Website
							\end{itemize}
						\item \textbf{BackingBean:}
					\end{itemize}
				
				\paragraph{authenticate.xhtml}
				KH\\
				\begin{itemize}
					\item \textbf{Beschreibung:}
					Auf dieser Seite kann man sich im System anmelden, ein neues Benutzerkonto generieren oder ein neues Passwort anfordern.
					\item \textbf{Links:} -
					\item \textbf{Buttons:}
						\begin{itemize}
							\item Registrieren: Durch Drücken dieses Buttons wird der neue Benutzer mit den eingegebenen Daten im System gespeichert und es wird eine Bestätigungsmail mit dem Verifizierungslink an die angegebene E-Mail-Adresse verschickt, sofern alle Daten korrekt eingegeben wurden.
							\item Anmelden: Durch Drücken dieses Buttons wird der registrierte Benutzer im System angemeldet und auf die Seite 'Meine Kurse' weitergeleitet, sofern Benutzername und Passwort korrekt eingegeben wurden.
							\item Neues Passwort anfordern: Durch Drücken dieses Buttons wird an die angegebene E-Mail-Adresse ein automatisch generiertes Passwort geschickt, sofern die Adresse im System existiert.
						\end{itemize}
					\item \textbf{Inputs:}
						\begin{itemize}
							\item Anrede (Registrierung): Hier wählt der Benutzer die Anrede 'Herr' oder 'Frau' aus.
							\item Vorname (Registrierung): Hier trägt der Benutzer seinen Vornamen ein.
							\item Name (Registrierung): Hier gibt der Benutzer seinen Namen ein.
							\item Benutzername (Registrierung): Hier gibt der Benutzer einen Benutzernamen ein.
							\item Passwort (Registrierung): Hier trägt der Benutzer ein Passwort ein.
							\item Passwort bestätigen (Registrierung): Hier gibt der Benutzer das gleiche Passwort erneut ein zur Bestätigung.
							\item Geburtstag (Registrierung): Hier gibt der Benutzer sein Geburtsdatum ein.
							\item Straße/Hausnummer (Registrierung): Hier gibt der Benutzer seine Straße und seine Hausnummer ein.
							\item Ort (Registrierung): Hier trägt der Benutzer seinen Ort ein.
							\item Postleitzahl (Registrierung): Hier trägt der Benutzer seine Postleitzahl ein.
							\item E-Mail-Adresse (Registrierung): Hier gibt der Benutzer seine E-Mail-Adresse ein.
							\item AGBs bestätigen (Registrierung): Durch Setzten des Häkchens bestätigt der Benutzer die AGBs. 
							\item Benutzername (Anmeldung): Der Benutzer gibt seinen Benutzernamen ein, mit dem er sich registriert hat.
							\item Passwort (Anmeldung): Der Benutzer gibt sein Passwort ein, mit dem er sich registriert hat.
							\item E-Mail-Adresse (Passwort vergessen): Der Benutzer gibt seine im System bereits gespeicherte E-Mailadresse ein.
						\end{itemize}
					\item \textbf{Outputs:} 
						\begin{itemize}
							\item Vorname Fehlermeldung (Registrierung): Ausgabe der Fehlermeldungen zu den Validatoren des Eingabefeldes.
							\item Name Fehlermeldung (Registrierung): Ausgabe der Fehlermeldungen zu den Validatoren des Eingabefeldes.
							\item Benutzername Fehlermeldung (Registrierung): Ausgabe der Fehlermeldungen zu den Validatoren des Eingabefeldes.
							\item Passwort Fehlermeldung (Registrierung): Ausgabe der Fehlermeldungen zu den Validatoren des Eingabefeldes.
							\item Passwort bestätigen Fehlermeldung (Registrierung): Ausgabe der Fehlermeldungen zu den Validatoren des Eingabefeldes.
							\item Geburtstag Fehlermeldung (Registrierung): Ausgabe der Fehlermeldungen zu den Validatoren des Eingabefeldes.
							\item Straße/Hausnummer Fehlermeldung (Registrierung): Ausgabe der Fehlermeldungen zu den Validatoren des Eingabefeldes.
							\item Ort Fehlermeldung (Registrierung): Ausgabe der Fehlermeldungen zu den Validatoren des Eingabefeldes.
							\item Postleitzahl Fehlermeldung (Registrierung): Ausgabe der Fehlermeldungen zu den Validatoren des Eingabefeldes.
							\item E-Mail-Adresse Fehlermeldung (Registrierung): Ausgabe der Fehlermeldungen zu den Validatoren des Eingabefeldes.
							\item AGBs bestätigen Fehlermeldung (Registrierung): Ausgabe der Fehlermeldungen zu den Validatoren des Eingabefeldes.
							\item Benutzername Fehlermeldung (Anmeldung): Ausgabe der Fehlermeldungen zu den Validatoren des Eingabefeldes.
							\item Passwort Fehlermeldung (Anmeldung): Ausgabe der Fehlermeldungen zu den Validatoren des Eingabefeldes.
							\item E-Mail-Adresse Fehlermeldung (Passwort vergessen): Ausgabe der Fehlermeldungen zu den Validatoren des Eingabefeldes.
						\end{itemize}
					\item \textbf{BackingBean:} authenticate.java
				\end{itemize}
				
				\paragraph{imprint.xhtml}
					KH\\
					\begin{itemize}
						\item \textbf{Beschreibung:} Auf dieser Seite kann das Impressum angesehen werden.
						\item \textbf{Links:} -
						\item \textbf{Buttons:} -
						\item \textbf{Inputs:} -
						\item \textbf{Outputs:} 
							\begin{itemize}
								\item	Anzeige des Impressums, welches vom Administrator festgelegt wurde.
							\end{itemize}
						\item \textbf{BackingBean:} -
					\end{itemize}
				
				\paragraph{agb.xhtml}
					RS\\
					\begin{itemize}
						\item \textbf{Beschreibung:} Dieses Facelet dient der Anzeige der Allgemeinen Geschäftsbedingungen.
						\item \textbf{Links:} -
						\item \textbf{Buttons:} -
						\item \textbf{Inputs:} -
						\item \textbf{Outputs:}
							\begin{itemize}
								\item von Betreibern festgelegte Allgemeine Geschäftsbedingungen
							\end{itemize}
						\item \textbf{BackingBean:} -
					\end{itemize}
				\paragraph{navigation.xhtml}
					RS\\
					\begin{itemize}
						\item \textbf{Beschreibung:} Hier kann, abhängig von der Benutzerrolle, durch die Seite navigiert werden.
						\item \textbf{Links:}
							\begin{itemize}
								\item Suche
								\item Sprache Deutsch
								\item Sprache Englisch
								\item Profil (Benutzer)
								\item Meine Kurse (Benutzer)
								\item Terminplaner (Benutzer)
							\end{itemize}
						\item \textbf{Buttons:}
							\begin{itemize}
								\item Logout (Benutzer, wenn eingeloggt)
								\item Anmelden (Anonym)
							\end{itemize}
						\item \textbf{Inputs:} -
						\item \textbf{Outputs:}
							\begin{itemize}
								\item persönliche Loginbenachrichtigung (Benutzer, wenn eingeloggt)
								\item Logo
							\end{itemize}
						\item \textbf{BackingBean:}
					\end{itemize}
		
			\subsubsection{courses}
				
				\paragraph{listCourses.xhtml}
					RS\\
					\begin{itemize}
						\item \textbf{Beschreibung:} Hier können alle Kurse angezeigt und durchsucht werden.
						\item \textbf{Links:} -
						\item \textbf{Buttons:}
							\begin{itemize}
								\item Anzeigen
								\item Suchen
							\end{itemize}
						\item \textbf{Inputs:}
							\begin{itemize}
								\item Angebotszeitraum
								\item Suchobjekt (abhängig von Benutzerrolle)
							\end{itemize}
						\item \textbf{Outputs:}
							\begin{itemize}
								\item Tabelle mit Ergebnissen der Suche
								\item Schaltfläche um zwischen Ergebnissen zu Blättern
							\end{itemize}
						\item \textbf{BackingBean:}
					\end{itemize}
				
				\paragraph{courseDetails.xhtml}
					RS\\
					\begin{itemize}
						\item \textbf{Beschreibung:} Detailanzeige eines einzelnen Kurses.
						\item \textbf{Links:}
						\item \textbf{Buttons:}
						\item \textbf{Inputs:}
						\item \textbf{Outputs:}
						\item \textbf{BackingBean:}
					\end{itemize}
		
		\subsection{users}
		
			\subsubsection{registeredUser}
				
				\paragraph{myCourses.xhtml}
					KH\\
					\begin{itemize}
						\item \textbf{Beschreibung:} Auf dieser Seite werden alle Kurse angezeigt, in die der Teilnehmer eingetragen ist.
						\item \textbf{Links:}
							\begin{itemize}
								\item Kurstitel: Über die Kurstitel gelangt der Benutzer zum auf die jeweilige Kursdetailseite.
							\end{itemize}
						\item \textbf{Buttons:}
						\item \textbf{Inputs:} -
						\item \textbf{Outputs:}
							\begin{itemize}
								\item Tabelle Auflistung aller Kurse: Hier werden dem Benutzer alle seine angemeldeten Kurse angezeigt.
							\end{itemize}
						\item \textbf{BackingBean:} myCourses.java
					\end{itemize}
				
				\paragraph{profile.xhtml}
					KH\\
					\begin{itemize}
						\item \textbf{Beschreibung:} Auf dieser Seite werden die persönlichen Daten und der Kontostand des Benutzers angezeigt. Der Benutzer kann die Daten ändern und sein Konto aufladen. Der Kursleiter kann hier den Benutzer aktivieren, und der Administrator den Benutzer löschen.
						\item \textbf{Links:} -
						\item \textbf{Buttons:}
							\begin{itemize}
								\item Bearbeiten: Nach Klicken auf diesen Button können die persönlichen Daten geändert werden. Der Button trägt nun die Aufschrift 'Speichern' und ist mit dessen dazugehöriger Methode hinterlegt.
								\item Speichern: Durch Drücken dieses Buttons werden die vorgenommenen Änderungen gespeichert, sofern alle Daten korrekt eingegeben wurden. Bei Änderung der E-Mail-Adresse wird außerdem eine Bestätigungsmail mit einem Verifizierungslink verschickt. Bei erfolgreicher Speicherung erscheint der Button 'Bearbeiten' anstelle des Button 'Speichern'.
								\item Durchsuchen: Durch Drücken dieses Buttons kann das eigene Dateiverzeichnis nach einem Bild durchsucht und hochgeladen werden.
								\item Konto aufladen: Durch Klicken dieses Buttons wird der Benutzer auf die Seite 'Kontoaufladung' weitergeleitet.
								\item Benutzer löschen: Durch Drücken dieses Buttons entfernt der Administrator diesen Benutzer aus dem System.
								\item Benutzer aktivieren: Durch diesen Button kann der Benutzer je nach Einstellung der Accountaktivierung vom Kursleiter oder vom Administrator aktiviert werden.
							\end{itemize}
						\item \textbf{Inputs:}
							\begin{itemize}
								\item Vorname: Hier kann der Benutzer seinen Vornamen ändern.
								\item Name: Hier kann der Benutzer seinen Namen ändern.
								\item Geburtstag: Hier kann der Benutzer sein Geburtsdatum ändern.
								\item Straße/Hausnummer: Hier kann der Benutzer seine Straße und Hausnummer ändern.
								\item Ort: Hier kann der Benutzer seinen Ort ändern.
								\item Postleitzahl: Hier kann der Benutzer seine Postleitzahl ändern.
								\item E-Mail-Adresse: Hier kann der Benutzer seine E-Mail-Adresse ändern.
								\item Benutzername: Hier kann der Benutzer seinen Benutzernamen ändern.
								\item Passwort: Hier kann der Benutzer sein Passwort ändern.
								\item Passwort bestätigen: Hier muss der Benutzer sein geändertes Passwort bestätigen.
								\item Benutzerrolle: Hier kann der Administrator die Benutzerrolle eines Nutzers ändern.
								\item Profilbild: Hier kann der Benutzer sein Profilbild ändern.
							\end{itemize}
						\item \textbf{Outputs:}
							\begin{itemize}
								\item Benutzer-ID: Ausgabe der automatisch generierten ID.
								\item Vorname Fehlermeldung: Ausgabe der Fehlermeldungen zu den Validatoren des Eingabefeldes.
								\item Name Fehlermeldung: Ausgabe der Fehlermeldungen zu den Validatoren des Eingabefeldes.
								\item Geburtstag Fehlermeldung: Ausgabe der Fehlermeldungen zu den Validatoren des Eingabefeldes.
								\item Straße/Hausnummer Fehlermeldung: Ausgabe der Fehlermeldungen zu den Validatoren des Eingabefeldes.
								\item Ort Fehlermeldung: Ausgabe der Fehlermeldungen zu den Validatoren des Eingabefeldes.
								\item Postleitzahl Fehlermeldung: Ausgabe der Fehlermeldungen zu den Validatoren des Eingabefeldes.
								\item E-Mail-Adresse Fehlermeldung: Ausgabe der Fehlermeldungen zu den Validatoren des Eingabefeldes.
								\item Benutzername Fehlermeldung: Ausgabe der Fehlermeldungen zu den Validatoren des Eingabefeldes.
								\item Passwort Fehlermeldung: Ausgabe der Fehlermeldungen zu den Validatoren des Eingabefeldes.
								\item Passwort bestätigen Fehlermeldung: Ausgabe der Fehlermeldungen zu den Validatoren des Eingabefeldes.
								\item Profilbild Fehlermeldung: Ausgabe der Fehlermeldungen zu den Validatoren des Eingabefeldes.
								\item Kontostand: Ausgabe des aktuellen Kontostandes.
								\item Tabelle Auflistung der Trainingskurse: Hier werden dem Kursleiter alle Kurse aufgelistet, die er leitet.
							\end{itemize}
						\item \textbf{BackingBean:} profile.java
					\end{itemize}
				
				\paragraph{buyCredits.xhtml}
					RS\\
					\begin{itemize}
						\item \textbf{Beschreibung:} Hier kann der Nutzer mittels Kreditkarte seinen systeminternen Kontostand erhöhen.
						\item \textbf{Links:} -
						\item \textbf{Buttons:}
							\begin{itemize}
								\item Konto aufladen
							\end{itemize}
						\item \textbf{Inputs:}
							\begin{itemize}
								\item Benutzer-ID
								\item Benutzername
								\item Nachname
								\item Vorname
								\item Kreditinstitut
								\item Kreditkartennummer
								\item Betrag
							\end{itemize}
						\item \textbf{Outputs:} -
						\item \textbf{BackingBean:}
					\end{itemize}
				
				\paragraph{scheduler.xhtml}
					RS\\
					\begin{itemize}
						\item \textbf{Beschreibung:} Persönlicher Terminplaner mit anstehenden Kursen.
						\item \textbf{Links:}
							\begin{itemize}
								\item Wochenansicht vorwärts: Zeigt die Termine der nächsten Woche an
								\item Wochenansicht rückwärts:
								Zeigt die Termine der letzten Woche an
							\end{itemize}
						\item \textbf{Buttons:} -
						\item \textbf{Inputs:} -
						\item \textbf{Outputs:}
							\begin{itemize}
								\item Wochentabelle: Zeigt von Montag bis Sonntag alle belegten Kurseinheiten an. Orientiert sich am klassischen Stundenplan, also mit stündlicher Ansicht.
							\end{itemize}
						\item \textbf{BackingBean:}
					\end{itemize}
				
				\paragraph{leaderProfile.xhtml}
					KH\\
					\begin{itemize}
						\item \textbf{Beschreibung:} Auf dieser Seite werden die Daten des Kursleiters, mit Ausnahme von sensiblen Daten wie Passwort oder Kontostand, und die von ihm geleiteten Kurse angezeigt.
						\item \textbf{Links:} -
						\item \textbf{Buttons:} -
						\item \textbf{Inputs:} -
						\item \textbf{Outputs:} 
							\begin{itemize}
								\item Tabelle Auflistung Kursleiterdaten: Hier werden die Daten des Kursleiters und die von ihm geleiteten Kurse angezeigt.
							\end{itemize}
						\item \textbf{BackingBean:} leaderProfile.java
					\end{itemize}
				
				\paragraph{listParticipants.xhtml}
					KH\\
					\begin{itemize}
						\item \textbf{Beschreibung:} Hier kann der registrierte Benutzer die Teilnehmer eines Kurses mit Benutzername und Profilbild ansehen. Dem Kursleiter werden zusätzlich die E-Mail-Adresse und die Information über den Erhalt von Kursnews angezeigt. Außerdem kann er einen Teilnehmer aus dem Kurs entfernen.
						\item \textbf{Links:} -
						\item \textbf{Buttons:}
							\begin{itemize}
								\item Löschen: Durch Drücken dieses Buttons kann der Kursleiter den entsprechenden Benutzer aus dem Kurs entfernen.
							\end{itemize}
						\item \textbf{Inputs:}
							 \begin{itemize}
							 	\item Entfernen: Durch Setzen des Häkchens wählt der Kursleiter diesen Benutzer aus, um ihn anschließend über den Button 'Löschen' zu entfernen.
							 \end{itemize}
						\item \textbf{Outputs:} -
						\item \textbf{BackingBean:} listParticipants.java
					\end{itemize}
			
			\subsubsection{courseLeader}
			
				\paragraph{editCourseUnit.xhtml}
					KH\\
					\begin{itemize}
						\item \textbf{Beschreibung:} Auf dieser Seite können Kursleiter beziehungsweise Administrator Kurseinheiten anlegen und bearbeiten, oder Kursteilnehmer hinzufügen und entfernen
						\item \textbf{Links:} -
						\item \textbf{Buttons:}
							\begin{itemize}
								\item Bearbeiten (Kurseinheit): Nach Klicken auf diesen Button können die Kursdaten geändert beziehungsweise eingetragen werden. Der Button trägt nun die Aufschrift 'Speichern' und ist mit dessen dazugehöriger Methode hinterlegt.
								\item Speichern (Kurseinheit): Durch Drücken dieses Buttons werden die vorgenommenen Änderungen gespeichert, sofern alle Daten korrekt eingegeben wurden. Bei erfolgreicher Speicherung erscheint der Button 'Bearbeiten' anstelle des Button 'Speichern'.
								\item Löschen (Kurseinheit): Durch Drücken des Button 'Löschen' wird die Kurseinheit entfernt.
								\item Löschen (Teilnehmer): Durch Drücken dieses Buttons wird der markierte Teilnehmer aus der Kurseinheit entfernt.
								\item Hinzufügen (Teilnehmer): Durch Drücken dieses Buttons wird der angegebene Teilnehmer zu dieser Kurseinheit hinzugefügt, sofern der Benutzername im System existiert und die Daten korrekt eingegeben wurden.
							\end{itemize}
						\item \textbf{Inputs:}
							\begin{itemize}
								\item Termin:
								\item Ort:
								\item Beschreibung:
								\item Preis:
								\item Kursleiter:
								\item Mindestteilnehmerzahl:
								\item Maximale Teilnehmerzahl:
								\item Teilnehmer markieren:
								\item Benutzer-ID:
								\item Name (Teilnehmer):
								\item Vorname (Teilnehmer):
							\end{itemize}
						\item \textbf{Outputs:}
							\begin{itemize}
								\item Termin (Fehlermeldung):
								\item Ort (Fehlermeldung):
								\item Beschreibung (Fehlermeldung):
								\item Preis (Fehlermeldung):
								\item Kursleiter (Fehlermeldung):
								\item Mindestteilnehmerzahl (Fehlermeldung):
								\item Maximale Teilnehmerzahl (Fehlermeldung):
								\item Benutzer-ID (Fehlermeldung):
								\item Name (Teilnehmer (Fehlermeldung)):
								\item Vorname (Teilnehmer (Fehlermeldung)):
							\end{itemize}
						\item \textbf{BackingBean:}
					\end{itemize}
				
				\paragraph{listUsers.xhtml}
					KH\\
					\begin{itemize}
						\item \textbf{Beschreibung:}
						\item \textbf{Links:}
						\item \textbf{Buttons:}
						\item \textbf{Inputs:}
						\item \textbf{Outputs:}
						\item \textbf{BackingBean:}
					\end{itemize}
			
			\subsubsection{systemAdministrator}
			
				\paragraph{adminManagement.xhtml}
					RS\\
					\begin{itemize}
						\item \textbf{Beschreibung:} Dieses Facelet stellt die zentrale Systemverwaltung für den Administrator dar. Hier kann er Kurse und Benutzer jeweils anlegen und verwalten, das Konto eines Nutzers aufladen, zu den Statistiken gelangen, den Überziehungskredit für Nutzer festlegen und das System optisch anpassen.
						\item \textbf{Links:} -
						\item \textbf{Buttons:}
							\begin{itemize}
								\item Benutzer verwalten: Navigiert zu Benutzer verwalten Seite
								\item Benutzer anlegen: Navigiert zu Benutzer anlegen Seite
								\item Kurse verwalten: Navigiert zu Kurse verwalten Seite
								\item Kurs anlegen: Navigiert zu Kurs anlegen Seite
								\item Aktivierungsmodalität speichern: Speichert die ausgewählte Art der Benutzeraktivierung
								\item Guthaben aufladen: Erhöht den Kontostand eines bestimmten Benutzer um die eingegebene Summe 
								\item Überziehungswert speichern: Speichert den Wert der möglichen Kontoüberziehung
								\item Statistiken anzeigen: Navigiert zur Anwendungsstatistik Seite
								\item Logo durchsuchen: Öffnet ein Fenster um das Logo der Website vom Computer auszuwählen
								\item Logo speichern: Speichert das benutzerdefinierte Logo für die Website
								\item Oberfläche CSS durchsuchen: Öffnet ein Fenster um die benutzerdefinierte CSS-Datei vom Computer auszuwählen
								\item Oberfläche CSS speichern: Speichert die benutzerdefinierte CSS-Datei
								\item Impressum bearbeiten: Navigiert zur Impressum bearbeiten Seite
							\end{itemize}
						\item \textbf{Inputs:}
							\begin{itemize}
								\item Accountaktivierung
								\item Benutzer-ID
								\item Benutzername
								\item Betrag
								\item Überziehungswert
							\end{itemize}
						\item \textbf{Outputs:} -
						\item \textbf{BackingBean:}
					\end{itemize}
				
				\paragraph{createUser.xhtml}
					RS\\
					\begin{itemize}
						\item \textbf{Beschreibung:} Hier kann der Systemadministrator einen neuen Benutzer mit festgelegter Benutzerrolle anlegen.
						\item \textbf{Links:} -
						\item \textbf{Buttons:}
							\begin{itemize}
								\item Durchsuchen - Profilbild
								\item Benutzer anlegen
							\end{itemize}
						\item \textbf{Inputs:}
							\begin{itemize}
								\item Anrede
								\item Vorname
								\item Nachname
								\item Geburtsdatum
								\item Straße
								\item Postleitzahl
								\item Ort
								\item Benutzername
								\item Passwort
								\item Passwort wiederholen
								\item E-Mail
								\item Benutzerrolle
							\end{itemize}
						\item \textbf{Outputs:}
							\begin{itemize}
								\item Benutzer-ID
								\item Profilbild
							\end{itemize}
						\item \textbf{BackingBean:}
					\end{itemize}
				
				\paragraph{createCourse.xhtml}
					KH\\
					\begin{itemize}
						\item \textbf{Beschreibung:}
						\item \textbf{Links:}
						\item \textbf{Buttons:}
						\item \textbf{Inputs:}
						\item \textbf{Outputs:}
						\item \textbf{BackingBean:}
					\end{itemize}
				
				\paragraph{editImprint.xhtml}
					KH\\
					\begin{itemize}
						\item \textbf{Beschreibung:}
						\item \textbf{Links:}
						\item \textbf{Buttons:}
						\item \textbf{Inputs:}
						\item \textbf{Outputs:}
						\item \textbf{BackingBean:}
					\end{itemize}
				
				\paragraph{statistics.xhtml}
					RS\\
					\begin{itemize}
						\item \textbf{Beschreibung:} Hier kann der Administrator die Statistiken zum System betrachten.
						\item \textbf{Links:}
						\item \textbf{Buttons:}
						\item \textbf{Inputs:}
						\item \textbf{Outputs:}
						\item \textbf{BackingBean:}
					\end{itemize}