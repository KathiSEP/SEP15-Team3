\chapter{ER-Modell}

\begin{tiny}
PC
\end{tiny}

\section{Diagramm}

In diesem Kapitel werden die systeminternen Entitäten und deren Relationen untereinander sowie alle dazugehörigen Attribute in einem ER-Diagramm dargestellt und kurz beschrieben.

\includegraphics[scale=0.085]{./Grafiken/ER-Diagramm.pdf}

\section{Beschreibung}
\subsection{User}
Ein im System registrierter Nutzer, welcher entweder als einfacher registrierter Nutzer, oder zusätzlich als Kursleiter, Administrator, oder auch beides gespeichert wird und anhand der 'ID' eindeutig identifizierbar ist. Ein User kann eine Adresse angeben (vgl. 'Address'). Zusätzlich werden die gezahlten Beiträge des Nutzers in der systeminternen Geldstatistik (vgl. 'Statistics') gespeichert.

\subsection{Course Instructor}
Ein registrierter Nutzer, welcher im System als Kursleiter gespeichert und anhand seiner 'ID' eindeutig identifizierbar ist. Ein 'Course instructor' kann beliebig viele Kurse leiten (vgl. 'Course').

\subsection{System admin}
Ein registrierter Nutzer, welcher im System als Systemadministrator gespeichert ist und anhand seiner 'ID' eindeutig identifizierbar ist. Ein 'System admin' kann sowohl beliebig viele Systemattribute (vgl. 'System attributes') editieren als auch beliebig die Gestaltung der Webanwendung anpassen (vgl. 'Customization data').

\subsection{Address}
Die Anschrift eines registrierten Nutzers bzw. einer Kurseinheit (vgl. 'Course unit'). Diese kann entweder genau einem 'User' oder genau einer 'Course unit' anhand der jeweiligen 'ID' eindeutig zugeordnet werden.

\subsection{System attributes}
Enthält vom Systemadministrator festgelegte Attribute, welche der Funktionalität der Webanwendung dienen. Diese können bei Bedarf von allen im System gespeicherten Administratoren editiert werden.

\subsection{Customization data}
Enthält den Titel der Webanwendung ('System title'), den Titel der CSS-Datei ('CSS') und kann bei Bedarf von allen im System gespeicherten Administratoren editiert werden.

\subsection{Course}
Ein angebotener Kurs zudem sich beliebig viele Nutzer anmelden können (vgl. 'Course participants') und von mindestens einem 'Course instructor' geleitet wird. Ein 'Course' kann beliebig viele Kurseinheiten (vgl. 'Course units') anbieten und ist anhand der 'ID' eindeutig identifizierbar.

\subsection{Course unit}
Eine Kurseinheit, welche innerhalb genau eines Kurses (vgl. 'Course') angeboten wird und anhand seiner 'ID' eindeutig identifizierbar ist. Sie hat genau eine Anschrift (vgl. 'Address'). Zu einer Kurseinheit können sich beliebig viele Nutzer anmelden (vgl. 'Course unit participants'), vorausgesetzt die maximale Teilnehmerzahl ('max. participants') ist noch nicht erreicht. Zusätzlich werden die gezahlten Beiträge einer Kurseinheit in der systeminternen Geldstatistik (vgl. 'Statistics') gespeichert.

\subsection{Statistics}
Die Geldstatistik, welche die über das System eingenommen Geldbeträge ('Amount') angibt.

\subsection{Course participant}
Ein zu mindestens einem Kurs (vgl. 'Course') angemeldeter Nutzer.

\subsection{Course unit participant}
Ein zu mindestens einer Kurseinheit (vgl. 'Course unit') angemeldeter Nutzer.