\chapter{Tests aus dem Pflichtenheft}

\begin{landscape}
	\section{Testfälle für den Systemadministrator ohne bestehenden Datensatz}	
		\begin{tabular}{|p{2.0cm} |p{5.0cm}|p{3.0cm}|p{5.0cm}|p{4.0cm}|p{4.0cm}|}
			\hline \textbf{Testnr.} & \textbf{Bezeichnung} & \textbf{Ergebnis} & \textbf{Ursache} & \textbf{Ergebnis} & \textbf{Ursache} \\ 
      	    \hline    T10-10   &      AuthenticationTest (zusätzlich werden Fehlerfälle geprüft, die bei der Anmeldung auftreten können)   &  erfolgreich &                  &                   &                  \\ 
			
			\hline T10-80, T10-90   &      CreateCourseTest  (Das Anlegen der Kurse aus den beiden Testfällen wird in einer Testklasse durchgeführt. Beide Kurse werden angelegt, zusätzlich wird auf weitere Fehlerfälle geprüft)              &      erfolgreich             &                  &                   &                  \\ 
			\hline       &          &          &        &         &       \\
			\hline 
		\end{tabular} \ \\
		\ \\
	\section{Testfälle für den Kursleiter}
		\begin{tabular}{|p{2.0cm} |p{5.0cm}|p{3.0cm}|p{5.0cm}|p{4.0cm}|p{4.0cm}|}
			\hline \textbf{Testnr.} & \textbf{Bezeichnung} & \textbf{Ergebnis} & \textbf{Ursache} & \textbf{Ergebnis} & \textbf{Ursache} \\
			\hline       &          &          &        &         &       \\
			\hline       &          &          &        &         &       \\
			\hline       &          &          &        &         &       \\
			\hline       &          &          &        &         &       \\
			\hline 
		\end{tabular} \ \\
		\ \\
				
	\section{Testfälle für anonymer Benutzer}
		\begin{tabular}{|p{2.0cm} |p{5.0cm}|p{3.0cm}|p{5.0cm}|p{4.0cm}|p{4.0cm}|}
			\hline \textbf{Testnr.} & \textbf{Bezeichnung} & \textbf{Ergebnis} & \textbf{Ursache} & \textbf{Ergebnis} & \textbf{Ursache} \\
			\hline   T30-30    & RegistrationTest (zustätzlich werden weitere Fehlerfälle geprüft, die bei der Registrierung auftreten können)   &   fehlgeschlagen       &    Die E-Mail-Adresse 'Katharina\_hoelzl@web.de wurde akzeptiert, obwohl katharina\_hoelzl@web.de bereits in der Datenbank vorhanden war. Verbesserung in der RegisterUserBean: Mail wird nun zu lower case konvertiert.    &    erfolgreich     &       \\
			\hline       &          &          &        &         &       \\
			\hline       &          &          &        &         &       \\
			\hline       &          &          &        &         &       \\
			\hline 
		\end{tabular} \ \\
		\ \\
			
	\section{Testfälle für den registrierten Benutzer}
		\begin{tabular}{|p{2.0cm} |p{5.0cm}|p{3.0cm}|p{5.0cm}|p{4.0cm}|p{4.0cm}|}
			\hline \textbf{Testnr.} & \textbf{Bezeichnung} & \textbf{Ergebnis} & \textbf{Ursache} & \textbf{Ergebnis} & \textbf{Ursache} \\
			\hline T40-10   &  UserNotYetActivatedTest        & erfolgreich   &        &         &       \\
			\hline T40-20   & AccountActivationByAdmin Test  & erfolgreich    &        &         &       \\	
			\hline T40-70   & SignUpForCourseYogaTest & erfolgreich &        &         &       \\	
			\hline T40-140  & HelpTest & erfolgreich &        &         &       \\	
			\hline T40-200  & LogoutTest & erfolgreich &        &         &       \\	
			\hline       &          &          &        &         &       \\
			\hline 
		\end{tabular} \ \\
		\ \\
			
	\section{Testfälle mit Datensatz}	
		\begin{tabular}{|p{2.0cm} |p{5.0cm}|p{3.0cm}|p{5.0cm}|p{4.0cm}|p{4.0cm}|}
			\hline \textbf{Testnr.} & \textbf{Bezeichnung} & \textbf{Ergebnis} & \textbf{Ursache} & \textbf{Ergebnis} & \textbf{Ursache} \\
			\hline  T40-180, T50-40  & ListParticipantsTest (Die beiden Testfälle 'Teilnehmer ansehen' und 'Teilnehmer aus Kurs entfernen' wurden in eine Testklasse zusammengefasst.) &     erfolgreich     &        &         &       \\
			\hline  T50-50  &  AddLeaderToCourseTest (zusätzlich wurden Fehlerfälle geprüft) &  fehlgeschlagen & Es wurde bei der Eingabe der KursleiterID nicht geprüft ob die ID zu einem Kursleiter gehört. Behebung des Fehlers durch hinzufügen des Validators CourseInstructor zum Facelet CourseDetail.xhtml  &  erfolgreich       &       \\
			\hline       &          &          &        &         &       \\
			\hline       &          &          &        &         &       \\
			\hline       &          &          &        &         &       \\
			\hline 
		\end{tabular} \ \\
		\ \\
	
\end{landscape}