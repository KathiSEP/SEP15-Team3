\chapter{Lasttests}

In diesem Abschnitt wurden Tests zur Belastbarkeit der Webanwendung durchgeführt. Dabei wurden die Ladezeiten des Systems 
geprüft, mit jeweils verschiedenen Anzahlen an parallel eingeloggten Benutzern und zur Verfügung gestellten Datenbankverbindungen.
Es wurde vier verschiedene Anzahlen an parallelen Sessions sowie zwei verschiedene Anzahlen an Datenbankverbindungen getestet. Dabei 
wurde von jeder Session über eine gemeinsame URL auf einen einzigen Server zugegriffen. Die Parallelität wurde unterteilt in 1, 5, 15 
und 20 gleichzeitig aktiven Benutzersessions. Die Tests wurden jeweils einmal mit zwei und einmal mit 10 zur Verfügung gestellten 
Datenbankverbindungen durchgeführt.

\section{Durchgeführte Tests}

\subsection{TestMeasurement}

\subsubsection{Beschreibung}

TestMeasurement beinhaltet, neben dem Ablauf, Variablen, welche die aktuelle Tageszeit auf dem Rechner in Millisekunden angibt, 
jeweils unmittelbar vor und nach einer zu testenden, durchgeführten Aktion. Durch Subtrahieren wird die Zeitspanne ermittelt und zur 
Auswertung auf der Konsole ausgegeben. Dieser Test wurde jeweils drei mal während den parallel laufenden Tests ausgeführt, um die 
verschiedenen Ladezeiten der jeweiligen Funktion zu messen.

\subsubsection{Ablauf}

Der Benutzer (in der Rolle eines Administrators) logged sich im System ein und führt eine Kurssuche durch, bei dem alle vorhandenen Kurse angezeigt werden. Der Kurs "Zweiter Test" wird angeklickt und auf dessen Detailseite meldet sich der Benutzer zu zwei Kurseinheiten ein, von denen er sich unmittelbar danach wieder abmeldet. Anschließend wird auf der Kurssuche-Seite nach dem Kurs "Yoga" geklickt und zu dessen Detailseite navigiert. Von dieser Seite geht der Benutzer über die Navigatinosleiste auf die Seite zur Seitenverwaltung und lässt sich bei der Nutzersuche alle registrierten Nutzer anzeigen. Daraufhin logged sich der Benutzer aus dem System aus.

\subsection{LoadTest}

\subsubsection{Beschreibung}

Der LoadTest wurde an bis zu 20 verschiedenen Rechnern durchgeführt. An jedem Rechner wurde genau eine .jar Datei ausgeführt, welche diesen Test beinhaltet. Dabei unterscheiden sich die Tests nur beim Beutzernamen und Passwort, um jeweils 1, 5, 15 und 20 
verschiedene Sessions zu simulieren.

\subsubsection{Ablauf}

Dieser Test beinhaltet zu einem Großteil den selben Ablauf wie TestMeasurement, es werden lediglich die Administratorfunktionen nicht durchgeführt, da der Benutzer keine Administratorrechte besitzt. Die Aktionen zwischen dem Login und Logout werden über eine for-Schleife 51 mal wiederholt.

