\chapter{Umgesetzte Wunschkriterien}
Im nachfolgenden Kapitel werden alle unsere Wunschkriterien und ihre Umsetzungen erläutert. \ \\

\begin{itemize}
	\item \textbf{Englisch als Sprachvariante:} \\Vorgesehen war von uns im Pflichtenheft, dass als Standardsprache Deutsch gesetzt ist. Von jeder Seite des Systems aus sollte der Benutzer die Sprache zu Englisch ändern können. In unserer Umsetzung werden alle Buttonbeschriftungen und alle ausgegebenen Messages in die jeweilige Sprache übersetzt. Des Weiteren werden alle Texte, die der Benutzer selbst nicht bearbeiten kann in der jeweiligen ausgewählten Sprache angezeigt. Benutzereingaben werden nicht übersetzt. Als zusätzliches Feature haben wir für unsere bayerischen Benutzer zusätzlich noch die Sprachauswahl 'Bayerisch' realisiert. Die gewünschte Sprache kann über das Anklicken der entsprechenden Flagge ausgewählt werden, sobald der Benutzer im System eingeloggt ist.
	\item \textbf{Terminplaner:}\\ Als weiteres Zusatzfeature haben wir den Terminplaner verwirklicht. Analog zur Angabe im Pflichtenheft kann der registrierte Benutzer seinen persönlichen Terminplaner einsehen. Dort werden alle Kurstermine angezeigt, zu denen der Benutzer eingetragen ist. Diese Termine werden wie vorgesehen automatisch im Terminplaner hinzugefügt, wenn sich der Benutzer zu einer Kurseinheit einträgt und ebenso automatisch wieder entfernt, wenn sich der Benutzer aus der Kurseinheit austrägt.
	\item \textbf{Einschränken des angezeigten Kursangebots:}\\ Unser letztes Wunschkriterium, das Einschränken der Anzeige des Kursangebotes, haben wir ebenfalls wie im Pflichtenheft festgelegt, realisiert. Auf der Suchen-Seite kann sich der Benutzer das Gesamte Kursangebot anzeigen lassen. Über eine Drop-Down-Liste kann er nun den Zeitraum auswählen, in dem das Kursangebot angezeigt werden soll. Zur Auswahl stehen das Wochenangebot und das Monatsangebot und die entsprechenden Kurse werden nach Betätigen der Schaltfläche 'Anzeigen' aufgelistet.
\end{itemize} 