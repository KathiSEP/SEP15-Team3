\documentclass[12pt,a4paper]{scrreprt}
\usepackage[utf8x]{inputenc}
\usepackage{ucs}
\usepackage{amsmath}
\usepackage{amsfonts}
\usepackage{amssymb}
\usepackage{graphicx}
\usepackage[left=2.00cm, right=2.50cm]{geometry}

\newcommand{\Arbeitspaket}[5]{	\begin{tabular}{|p{4cm}|p{3cm}|p{3cm}|p{3cm}|p{3cm}|}
		\hline \textbf{Paketname} & \textbf{Geplanter Startzeitpunkt} & \textbf{Geplanter Endzeitpunkt} & \textbf{Geplanter Aufwand in h} & \textbf{Implementierer} \\ 
		\hline #1 & #2  & #3 & #4 & #5 \\ 
		\hline 
	\end{tabular} \ \\
	\ \\}


\begin{document}
	\paragraph{Vorbemerkung:}\ \\
	 \ \\
	{\it Dies ist lediglich ein Entwurfsdokument!!! \\ \\
		Die einzelnen Arbeitspakte sind, abgesehen von den Paketen die Basisfunktionen darstellen(z.b. DatabaseConnectionManager) nach Features gegliedert.\\
		Die Unterpunkte von Arbeitspaketen stellen die zu realisierenden Komponenten dar, welche nötig sind, damit das entsprechende Feature des Systems funktioniert. Die Unterpunkte sind, wenn möglich, \textbf{vertikal} orientiert.\ \\
		Die Reihenfolge der Arbeitspakete sollte in etwa die zu implementierende Reihenfolge darstellen um  Wartezeiten durch Abhängigkeiten zu minimieren(deshalb Basisfunktionalität zu Beginn).\ \\
		\ \\
		Die durchzuführenden Tests sind noch \textbf{nicht} aufgeführt. 
		 }
	\chapter*{Milestone 1}
	\Arbeitspaket{SessionUserBean.java}{b}{arg3}{arg4}{}
	
	
	\ \\
	\ \\
	\Arbeitspaket{Enumerationsklassen}{arg2}{arg3}{arg4}{arg5}
	\textbf{Unterpunkte:}
	\begin{itemize}
		\item UserStatus.java\\
			\begin{tabular}{|p{4cm}|p{4cm}|p{4cm}|}
				\hline Startzeitpunkt & Endzeitpunkt & Aufwand in h \\ 
				\hline &      &  \\ 
				\hline 
				\end{tabular}
		\item UserRole.java	\\
		\begin{tabular}{|p{4cm}|p{4cm}|p{4cm}|}
			\hline Startzeitpunkt & Endzeitpunkt & Aufwand in h \\ 
			\hline &      &  \\ 
			\hline 
		\end{tabular}
	\end{itemize}
	
	
	\ \\
	\ \\
	\Arbeitspaket{PropertyManager}{arg2}{arg3}{arg4}{arg5}
	\textbf{Unterpunkte:}
	\begin{itemize}
		\item PropertyManager.java\\
		\begin{tabular}{|p{4cm}|p{4cm}|p{4cm}|}
			\hline Startzeitpunkt & Endzeitpunkt & Aufwand in h \\ 
			\hline &      &  \\ 
			\hline 
		\end{tabular}
		\item ofCourse.properties\\
		\begin{tabular}{|p{4cm}|p{4cm}|p{4cm}|}
			\hline Startzeitpunkt & Endzeitpunkt & Aufwand in h \\ 
			\hline &      &  \\ 
			\hline 
		\end{tabular}	
	\end{itemize}
	
	
	\ \\
	\ \\
	\Arbeitspaket{EncodingFilter.java}{arg2}{arg3}{arg4}{arg5}
		
	
	
	\ \\
	\ \\
	\Arbeitspaket{DatabaseConnection- Manager.java}{arg2}{arg3}{arg4}{Tobias Fuchs}
	
	
	\ \\
	\ \\
	\Arbeitspaket{CheckPhase.java}{arg2}{arg3}{arg4}{Tobias Fuchs}
	
	\ \\
	\ \\
	\Arbeitspaket{Logging}{arg2}{arg3}{arg4}{}
	
	\ \\
	\ \\
	\Arbeitspaket{Mailversand}{arg2}{arg3}{arg4}{}
	\textbf{Unterpunkte:}
	\begin{itemize}
		\item MailBean.java\\
		\begin{tabular}{|p{4cm}|p{4cm}|p{4cm}|}
			\hline Startzeitpunkt & Endzeitpunkt & Aufwand in h \\ 
			\hline &      &  \\ 
			\hline 
		\end{tabular}
		\item MailingException.java	\\
		\begin{tabular}{|p{4cm}|p{4cm}|p{4cm}|}
			\hline Startzeitpunkt & Endzeitpunkt & Aufwand in h \\ 
			\hline &      &  \\ 
			\hline 
		\end{tabular}
		\item mail.properties\\
		\begin{tabular}{|p{4cm}|p{4cm}|p{4cm}|}
			\hline Startzeitpunkt & Endzeitpunkt & Aufwand in h \\ 
			\hline &      &  \\ 
			\hline 
		\end{tabular}
	\end{itemize}
		
	\ \\
	\ \\
	\Arbeitspaket{Datenbank Setup}{arg2}{arg3}{arg4}{arg5}
	\textbf{Unterpunkte:}
	\begin{itemize}
		\item DatabaseTableCreator.java\\
		\begin{tabular}{|p{4cm}|p{4cm}|p{4cm}|}
			\hline Startzeitpunkt & Endzeitpunkt & Aufwand in h \\ 
			\hline &      &  \\ 
			\hline 
		\end{tabular}
		\item SetupAdmin.java\\
		\begin{tabular}{|p{4cm}|p{4cm}|p{4cm}|}
			\hline Startzeitpunkt & Endzeitpunkt & Aufwand in h \\ 
			\hline &      &  \\ 
			\hline 
		\end{tabular}
		\item InvalidDBTransferException.java\\
		\begin{tabular}{|p{4cm}|p{4cm}|p{4cm}|}
			\hline Startzeitpunkt & Endzeitpunkt & Aufwand in h \\ 
			\hline &      &  \\ 
			\hline 
		\end{tabular}
	\end{itemize}
		
	\ \\
	\ \\
	\Arbeitspaket{Transaktion}{arg2}{arg3}{arg4}{}
	\textbf{Unterpunkte:}
		\begin{itemize}
			\item Transaction.java\\
			\begin{tabular}{|p{4cm}|p{4cm}|p{4cm}|}
				\hline Startzeitpunkt & Endzeitpunkt & Aufwand in h \\ 
				\hline &      &  \\ 
				\hline 
			\end{tabular}
			\item Connection.java	\\
			\begin{tabular}{|p{4cm}|p{4cm}|p{4cm}|}
				\hline Startzeitpunkt & Endzeitpunkt & Aufwand in h \\ 
				\hline &      &  \\ 
				\hline 
			\end{tabular}
		\end{itemize}
	
	\ \\
	\ \\
	\Arbeitspaket{DTO anlegen}{arg2}{arg3}{arg4}{a}
	\textbf{Unterpunkte:}
		\begin{itemize}
			\item User.java\\
			\begin{tabular}{|p{4cm}|p{4cm}|p{4cm}|}
				\hline Startzeitpunkt & Endzeitpunkt & Aufwand in h \\ 
				\hline &      &  \\ 
				\hline 
			\end{tabular}
			\item Course.java\\
			\begin{tabular}{|p{4cm}|p{4cm}|p{4cm}|}
				\hline Startzeitpunkt & Endzeitpunkt & Aufwand in h \\ 
				\hline &      &  \\ 
				\hline 
			\end{tabular}
			\item CourseUnit.java\\
			\begin{tabular}{|p{4cm}|p{4cm}|p{4cm}|}
				\hline Startzeitpunkt & Endzeitpunkt & Aufwand in h \\ 
				\hline &      &  \\ 
				\hline 
			\end{tabular}
			\item Address.java\\
			\begin{tabular}{|p{4cm}|p{4cm}|p{4cm}|}
				\hline Startzeitpunkt & Endzeitpunkt & Aufwand in h \\ 
				\hline &      &  \\ 
				\hline 
			\end{tabular}
			\item PaginationData.java\\
			\begin{tabular}{|p{4cm}|p{4cm}|p{4cm}|}
				\hline Startzeitpunkt & Endzeitpunkt & Aufwand in h \\ 
				\hline &      &  \\ 
				\hline 
			\end{tabular}
			\item SmtpServer.java\\
			\begin{tabular}{|p{4cm}|p{4cm}|p{4cm}|}
				\hline Startzeitpunkt & Endzeitpunkt & Aufwand in h \\ 
				\hline &      &  \\ 
				\hline 
			\end{tabular}
			\item Cycle.java\\
			\begin{tabular}{|p{4cm}|p{4cm}|p{4cm}|}
				\hline Startzeitpunkt & Endzeitpunkt & Aufwand in h \\ 
				\hline &      &  \\ 
				\hline 
			\end{tabular}
		\end{itemize}
	\ \\
	\ \\
	\Arbeitspaket{Footer}{arg2}{arg3}{arg4}{arg5}
	\textbf{Unterpunkte:}
	\begin{itemize}
		\item footer.xhtml\\
		\begin{tabular}{|p{4cm}|p{4cm}|p{4cm}|}
			\hline Startzeitpunkt & Endzeitpunkt & Aufwand in h \\ 
			\hline &      &  \\ 
			\hline 
		\end{tabular}
		\item FooterBean.java\\
		\begin{tabular}{|p{4cm}|p{4cm}|p{4cm}|}
			\hline Startzeitpunkt & Endzeitpunkt & Aufwand in h \\ 
			\hline &      &  \\ 
			\hline 
		\end{tabular}
	\end{itemize}
	
	\ \\
	\ \\
	\Arbeitspaket{Navigation}{arg2}{arg3}{arg4}{arg5}
	\textbf{Unterpunkte:}
	\begin{itemize}
		\item naviagtion.xhtml\\
		\begin{tabular}{|p{4cm}|p{4cm}|p{4cm}|}
			\hline Startzeitpunkt & Endzeitpunkt & Aufwand in h \\ 
			\hline &      &  \\ 
			\hline 
		\end{tabular}
		\item NavigationBean.java\\
		\begin{tabular}{|p{4cm}|p{4cm}|p{4cm}|}
			\hline Startzeitpunkt & Endzeitpunkt & Aufwand in h \\ 
			\hline &      &  \\ 
			\hline 
		\end{tabular}
	\end{itemize}
	
	
	\ \\
	\ \\
	\Arbeitspaket{LaunchSystem.java}{arg2}{arg3}{arg4}{a}


	\ \\
	\ \\
	\Arbeitspaket{PasswordHash.java}{arg2}{arg3}{arg4}{arg5}
	
	\ \\
	\ \\
	\Arbeitspaket{Anmeldung}{arg2}{arg3}{arg4}{Tobias Fuchs}
	\textbf{Unterpunkte:}
	\begin{itemize}
		\item authenticate.xhtml\\
		\begin{tabular}{|p{4cm}|p{4cm}|p{4cm}|}
			\hline Startzeitpunkt & Endzeitpunkt & Aufwand in h \\ 
			\hline &      &  \\ 
			\hline 
		\end{tabular}
		\item AuthenticateUserBean.java	\\
		\begin{tabular}{|p{4cm}|p{4cm}|p{4cm}|}
			\hline Startzeitpunkt & Endzeitpunkt & Aufwand in h \\ 
			\hline &      &  \\ 
			\hline 
		\end{tabular}
		\item UserDAO.getUser(Transaction trans, String name)\\
		\begin{tabular}{|p{4cm}|p{4cm}|p{4cm}|}
			\hline Startzeitpunkt & Endzeitpunkt & Aufwand in h \\ 
			\hline &      &  \\ 
			\hline 
		\end{tabular}
	\end{itemize}
	
	\ \\
	\ \\
	\Arbeitspaket{Registrierung}{arg2}{arg3}{arg4}{a}
	\textbf{Unterpunkte:}
	\begin{itemize}
		\item RegisterUserBean.java	\\
		\begin{tabular}{|p{4cm}|p{4cm}|p{4cm}|}
			\hline Startzeitpunkt & Endzeitpunkt & Aufwand in h \\ 
			\hline &      &  \\ 
			\hline 
		\end{tabular}
		\item UserDAO.createUser(Transaction trans, User user)\\
		\begin{tabular}{|p{4cm}|p{4cm}|p{4cm}|}
			\hline Startzeitpunkt & Endzeitpunkt & Aufwand in h \\ 
			\hline &      &  \\ 
			\hline 
		\end{tabular}
		\item UserNameValidator.java\\
		\begin{tabular}{|p{4cm}|p{4cm}|p{4cm}|}
			\hline Startzeitpunkt & Endzeitpunkt & Aufwand in h \\ 
			\hline &      &  \\ 
			\hline 
		\end{tabular}
		\item ConfirmPasswordValidator.java\\
		\begin{tabular}{|p{4cm}|p{4cm}|p{4cm}|}
			\hline Startzeitpunkt & Endzeitpunkt & Aufwand in h \\ 
			\hline &      &  \\ 
			\hline 
		\end{tabular}
		\item DateOfBirthValidator.java\\
		\begin{tabular}{|p{4cm}|p{4cm}|p{4cm}|}
			\hline Startzeitpunkt & Endzeitpunkt & Aufwand in h \\ 
			\hline &      &  \\ 
			\hline 
		\end{tabular}
	\end{itemize}
	
	
	\ \\	
	\ \\
	\Arbeitspaket{Passwort vergessen}{arg2}{arg3}{arg4}{a}
	\textbf{Unterpunkte:}
	\begin{itemize}
		\item LostPasswordBean.java	\\
		\begin{tabular}{|p{4cm}|p{4cm}|p{4cm}|}
			\hline Startzeitpunkt & Endzeitpunkt & Aufwand in h \\ 
			\hline &      &  \\ 
			\hline 
		\end{tabular}
		\item EMailValidator.java\\
		\begin{tabular}{|p{4cm}|p{4cm}|p{4cm}|}
			\hline Startzeitpunkt & Endzeitpunkt & Aufwand in h \\ 
			\hline &      &  \\ 
			\hline 
		\end{tabular}
		\item UserDAO.updateUser(Transaction trans, User user)\\
		\begin{tabular}{|p{4cm}|p{4cm}|p{4cm}|}
			\hline Startzeitpunkt & Endzeitpunkt & Aufwand in h \\ 
			\hline &      &  \\ 
			\hline 
		\end{tabular}
	\end{itemize}
		
		
   	\ \\
   	\ \\
   	\Arbeitspaket{Meine Kurse}{arg2}{arg3}{arg4}{Tobias Fuchs}
   	\textbf{Unterpunkte:}
	   	\begin{itemize}
    		\item myCourses.xhtml\\
    		\begin{tabular}{|p{4cm}|p{4cm}|p{4cm}|}
    			\hline Startzeitpunkt & Endzeitpunkt & Aufwand in h \\ 
    			\hline &      &  \\ 
    			\hline 
    		\end{tabular}	
    		\item MyCoursesBean.java\\
    		\begin{tabular}{|p{4cm}|p{4cm}|p{4cm}|}
    			\hline Startzeitpunkt & Endzeitpunkt & Aufwand in h \\ 
    			\hline &      &  \\ 
    			\hline 
    		\end{tabular}
    		\item CourseDAO.getCoursesOf(Transaction trans, int userID)\\
    		\begin{tabular}{|p{4cm}|p{4cm}|p{4cm}|}
    			\hline Startzeitpunkt & Endzeitpunkt & Aufwand in h \\ 
    			\hline &      &  \\ 
    			\hline 
    		\end{tabular}
    	\end{itemize}
	
	\ \\
	\ \\
	\Arbeitspaket{Kurssuche}{arg2}{arg3}{arg4}{a}
	\textbf{Unterpunkte:}
	\begin{itemize}
		\item search.xhtml\\
		\begin{tabular}{|p{4cm}|p{4cm}|p{4cm}|}
			\hline Startzeitpunkt & Endzeitpunkt & Aufwand in h \\ 
			\hline &      &  \\ 
			\hline 
		\end{tabular}	
		\item SearchCourseBean.java\\
		\begin{tabular}{|p{4cm}|p{4cm}|p{4cm}|}
			\hline Startzeitpunkt & Endzeitpunkt & Aufwand in h \\ 
			\hline &      &  \\ 
			\hline 
		\end{tabular}
		\item CourseDAO.getCourses(Transaction trans, PaginationData pagination)\\
		\begin{tabular}{|p{4cm}|p{4cm}|p{4cm}|}
			\hline Startzeitpunkt & Endzeitpunkt & Aufwand in h \\ 
			\hline &      &  \\ 
			\hline 
		\end{tabular}
		\item CourseDAO.getCourses(Transaction trans,PaginationData pagination, String searchString)\\
		\begin{tabular}{|p{4cm}|p{4cm}|p{4cm}|}
			\hline Startzeitpunkt & Endzeitpunkt & Aufwand in h \\ 
			\hline &      &  \\ 
			\hline 
		\end{tabular}
		\item CourseDAO.getCoursesOrdered(Transaction trans, PaginationData pagination, String searchString, String orderParam)\\
		\begin{tabular}{|p{4cm}|p{4cm}|p{4cm}|}
			\hline Startzeitpunkt & Endzeitpunkt & Aufwand in h \\ 
			\hline &      &  \\ 
			\hline 
		\end{tabular}
	\end{itemize}
	
	\Arbeitspaket{Benutzerprofil anzeigen}{arg2}{arg3}{arg4}{arg5}
	\textbf{Unterpunkte:}
	\begin{itemize}
		\item profile.xhtml\\
		\begin{tabular}{|p{4cm}|p{4cm}|p{4cm}|}
			\hline Startzeitpunkt & Endzeitpunkt & Aufwand in h \\ 
			\hline &      &  \\ 
			\hline 
		\end{tabular}
		\item UserProfileBean.java\\
		\begin{tabular}{|p{4cm}|p{4cm}|p{4cm}|}
			\hline Startzeitpunkt & Endzeitpunkt & Aufwand in h \\ 
			\hline &      &  \\ 
			\hline 
		\end{tabular}
		\item UserDAO.getUser(Transaction trans, int userID)\\
		\begin{tabular}{|p{4cm}|p{4cm}|p{4cm}|}
			\hline Startzeitpunkt & Endzeitpunkt & Aufwand in h \\ 
			\hline &      &  \\ 
			\hline 
		\end{tabular}
	\end{itemize}

	\chapter*{Milestone 2}
	
	\chapter*{Milestone 3}
	
\end{document}}{Nenner}