\newcommand{\kursiv}[1]{{\it #1}}
\chapter{Implementierungsplan}
Die Milestones sind so geplant, dass sie jeweils eine zeitliche Dauer von 5 Tagen haben. Die Arbeitspakete werden
je nach Dringlichkeit in diese drei Milestones eingeteilt. Die einzelnen Implementierer werden jeweils
mit einer Gesamtstundenanzahl von maximal 30 Stunden pro Milestone eingeplant. Die Aufgabenpakete wurden, soweit dies möglich war, vertikal zusammengestellt.
Dies bedeutet, dass beispielsweise ein Bearbeiter das Facelet, die dazu
gehörige Businesslogik und die benötigten DAOs erstellt.\\
\ \\
Sollten die geplanten Zeiten aufgrund unerwarteter Ereignisse oder Problemen bei der Implementierung nicht eingehalten werden können, so werden Pakete aus dem
ersten bzw. zweiten Milstone in den dritten Milestone verschoben. Der dritte Milestone ist deshalb bisher auch mit etwas weniger Umfang geplant.
Sollte dies von Nöten sein, so werden die Wunschkriterien, also der Terminplaner und die Unterstützung von Mehrsprachigkeit nicht umgesetzt.\\
Grundsätzlich wird aber davon ausgegangen alle Wunschkriterien umzusetzen.
\section{Milestone 1}

Der Inhalt des ersten Milestones besteht einerseits aus den kritischen Arbeitspakete und zum Anderen aus
ersten Grundfunktionalitäten.\\
Zu den kritischen Arbeitspaketen zählt der \kursiv{DatabaseConnectionManager}. Dieser ist zuständig für das etablieren der Datenbankverbindungen und die Weiterverteilung dieser. Da viele andere Komponenten eine Datenbankverbindung benötigen und somit davon abhängen, muss dieser zu Beginn der Implementierung realisiert werden.\\
Außerdem wird wird der \kursiv{PropertyManager} implementiert, welcher dafür zuständig ist, die nötigen Parameter aus der Konfigurationsdatei zu laden.
Eine weitere wichtige Aufgabe, die bereits zum ersten Milestone zugeordnet wird, ist die
Implementierung des Systemstarts. Dadurch wird sichergestellt, dass bei Systemstart alle
benötigten Initialisierungen gestartet werden, wie das Einlesen der Propertydateien, die Erstellung der Datenbanktabellen und Initialisierung des \kursiv{DatabaseConnectionManagers}.
Des weiteren ist das Aufsetzen der Datenbank, d.h. die Erstellung der benötigten Tabellen, ... und das zur Verfügung stellen des Logging - Mechanismus Teil des ersten Milestones.\\
Darüber hinaus wird der  UTF-8-Filter implementiert, welcher sicherstellt, dass die Benutzereingaben
im richtigen Encoding in der Datenbank hinterlegt werden.\\
Als letzter Punkt der kritischen Grundfunktionalität wird hier die Implementierung der automatischen E-Mail-Generierung vorgenommen, welche für die Versendung von E-Mails, wie etwa die Bestätigungs-E-Mails bei der Registrierung, zuständig ist.\\ 
Zusätzlich zu den kritischen Komponenten werden im ersten Milestone auch erste Grundfunktionalitäten des Systems implementiert, d.h. die Struktur der View wird implementiert und DTOs werden angelegt.\\
Der erste Milestone beinhaltet ebenfalls die Implementierung des Anmeldens, der Registrierung und der Passwort vergessen - Funktion.\\
Außerdem soll die \kursiv{Meine Kurse} - Seite angezeigt werden können und die Suche nach Kursen möglich sein.
Ebenfalls werden im ersten Milestone erste Tests durchgeführt um Fehler aufzudecken.\\

\section{Milestone 2}

Der zweite Milestone beinhaltet die Implementierung des Benutzerprofils, sowie die Edit-
Funktionalität. Dies beinhaltet, dass das benötigte Facelet und die Logik dazu, sowie die
benötigten DAO-Methoden implementiert werden.\\
Außerdem wird die Möglichkeit zur Verfügung gestellt einen Benutzer manuell anzulegen oder
Benutzer endgültig aus dem System zu löschen.
Ein weiterer Bestandteil ist die Implementierung der Detailansicht der Kurse sowie deren Edit-Funktionalität.
Außerdem wird die Anmeldung zu Kursen und zu Kurseinheiten realisiert. Im Zuge dessen natürlich auch die Abmeldefunktion von Kursen bzw. Kurseinheiten.
Ein weiterer Bestandteil dieses Milestones ist die Funktionalität einen Kurs anlegen zu können bzw. diesen wieder zu löschen. 
Außerden wird die Erstellung, Bearbeitung und Löschung von Kurseinheiten implementiert. Auch die Bearbeitung von regelmäßigen Kurseinheiten in einem Zug wird realisiert.
Um Fehler möglichst früh aufzudecken, werden auch hier wieder erste Tests implementiert
und Codereview durchgeführt.


\section{Milestone 3}

Der dritte Milestone beinhaltet die Implementierung der noch fehlenden Funktionalitäten und der Wunschkriterien.
Zur noch fehlenden Funktionalität gehört die Benutzersuche, d.h. im Detail das zugehörige Facelet, dessen Logik und dies entsprechenden DAO - Methoden. \\
Des weiteren wird die Funktionalität für den Kursleiter oder Administrator Benutzer zu aktivieren implementiert. \\
Außerdem wird die Anzeige der Teilnehmerliste von Kursen realisiert. Darüber hinaus wird noch das Feature implementiert sich das eingeschränkte Profil des Kursleiters eines Kurses anzeigen zu lassen. \\
Zusätzlich wird noch die Administratorseite realisiert, welche die  Einstellung bezüglich des Systems, wie etwa die Festlegung des Überziehungskredits für dem Administrator zu Verfügung stellt. 
Ebenfalls werden im dritten Milestone die CustomExceptionHandler.java- und die
CustomExceptionHandlerFactory.java-Klasse implementiert, sowie die zugehörige Fehlerseite.
\ \\
In diesem Milestone werden zusätzlich die Wunschkriterien umgesetzt. Diese beinhalten
die Mehrsprachigkeit des Systems, sowie den persönlichen Terminplaner des Benutzers in dem alle seine angemeldeten Kurseinheiten aufgeführt werden.
Außerdem werden hier die Hilfe-, Impressums- und AGB-Seite umgesetzt.
Es erfolgen außerdem Tests der Systemfunktionalität.

