\chapter{Datenbankschema}

\begin{tiny}
PC
\end{tiny}

In diesem Kapitel werden die Tabellen der Datenbank des Systems \grqq ofCourse\grqq\ beschrieben und die Eigenschaften des Schemas erläutert.

\section{Namenskonventionen}
Zum besseren Verständnis sowie zur einfacheren Wartbarkeit und Erweiterung des Datenbankschemas werden die SQL-Syntax Namenskonventionen eingehalten. Dazu werden in den SQL-Statements Schlüsselwörter nur in Großbuchstaben geschrieben. Die Namen der Tabellen sowie deren Attribute bestehen ausschließlich aus Kleinbuchstaben, Wörter werden dabei durch einen Unterstrich getrennt. Die Benennung der Tabellen \texttt{users}, \texttt{courses}, \texttt{course\_units}, \texttt{addresses} und \texttt{cycles} ergibt sich aus dem Plural der Bezeichnungen der entsprechenden DTOs aus dem Paket model (vgl. Klassendiagramm). Die Benennung der Tabellen \texttt{system\_attributes} und \texttt{customization\_data} ergibt sich aus der jeweiligen Entität des im Entwurf vorgestellten ER-Diagramms.

\section{Datenbanktabellen}
In diesem Abschnitt werden die Datenbanktabellen aufgelistet und beschrieben. Es wird die zweite Normalform verwendet. Zu jeder Tabelle wird das SQL CREATE-Statement angegeben, welches die jeweilige Tabelle erzeugt. Des Weiteren werden für bestimmte Spalten Enums verwendet (vgl. 6.4).

\subsection{users}
Die Tabelle \texttt{users} enthält sämtliche im System registrierten Nutzer. Dazu zählen ebenso Kursleiter als auch Administratoren. Verschiedene Nutzer können nicht dieselbe E-Mail Adresse oder den gleichen Nickname teilen. Dies wird durch Angeben des Schlüsselworts \texttt{UNIQUE} sichergestellt. Des Weiteren werden nur die Hashwerte der Passwörter gespeichert.
	
\begin{verbatim}

CREATE TABLE users (
    id                     SERIAL          PRIMARY KEY,
    first_name             VARCHAR(25),
    name                   VARCHAR(25),
    nickname               VARCHAR(25)     NOT NULL UNIQUE,
    e-mail                 VARCHAR(319)    NOT NULL UNIQUE,
    pw_hash                VARCHAR(64)     NOT NULL,
    date_of_birth          DATE,
    gender                 GENDER,
    credit_balance         BOOLEAN         NOT NULL,
    e-mail_verification    BOOLEAN         NOT NULL,
    admin_verfication      BOOLEAN         NOT NULL,
    profile_image          BYTEA,
    user_role              ROLE            NOT NULL,
)
\end{verbatim}

\subsection{courses}
Die Tabelle \texttt{courses} enthält alle im System angebotenen Kurse. Die Spalte \texttt{user\_to\_be\_informed} bezieht sich auf das Attribut \texttt{id} der Tabelle \texttt{users} und beinhaltet die Nutzer, die automatisch über Kursinformationen zum entsprechenden Kurs benachrichtigt werden sollen. Falls die maximale Anzahl an Teilnehmern für einen Kurs angegeben wurde wird zusätzlich wird geprüft, ob dieser Wert größer Null ist.

\begin{verbatim}
CREATE TABLE courses (
    id                     SERIAL     PRIMARY KEY,
    user_to_be_informed    REFERENCES users(id)    ON DELETE CASCADE,
    titel                  TEXT(100),
    max_participants       INTEGER    CHECK (max_participants > 0),
    start_date             DATE       NOT NULL,
    end_date               DATE       NOT NULL,
    image                  BYTEA,
)
\end{verbatim}

\subsection{course\_units}
Die Tabelle \texttt{course\_units} enthält alle Kurseinheiten, die innerhalb eines Kurs angeboten werden. Dieser wird in der Spalte \texttt{course\_id} referenziert. Die maximale und minimale Anzahl an Teilnehmern einer muss größer Null sein. Sollte eine Kurseinheit kostenpflichtig sein wird geprüft, ob der angegeben Betrag (\texttt{fee}) größer oder gleich Null ist.

\begin{verbatim}
CREATE TABLE course_units (
    id                  SERIAL       PRIMARY KEY,
    course_id           REFERENCES courses(id)   ON DELETE CASCADE,
    max_participants    INTEGER      NOT NULL    CHECK (max_participants > 0),
    titel               TEXT(100),
    min_participants    INTEGER                  CHECK (min_participants > 0),
    fee                 DECIMAL(6,2)             CHECK (fee >= 0),
    location            TEXT(100),
    start_time          TIMESTAMP    NOT NULL,
    end_time            TIMESTAMP    NOT NULL,
    description         TEXT(1000)
)
\end{verbatim}

\subsection{addresses}
Die Tabelle \texttt{addresses} enthält alle Adressen der registrierten Nutzer sowie der angebotenen Kurseinheiten. Diese werden in den Spalten \texttt{user} und \texttt{course\_unit} referenziert.

\begin{verbatim}
CREATE TABLE addresses (
    id             SERIAL       PRIMARY KEY,
    user           REFERENCES users(id)        ON DELETE CASCADE,
    course_unit    REFERENCES course_units(id) ON DELETE CASCADE,
    country        VARCHAR(15)  NOT NULL,
    city           VARCHAR(25)  NOT NULL,
    zip_code       INTEGER      NOT NULL,
    street         VARCHAR(35),
    house_nr       INTEGER
)
\end{verbatim}

\subsection{cycles}
Die Tabelle \texttt{cycles} enthält alle Zyklen innerhalb der angebotenen Kurse, welche in der Spalte \texttt{course} referenziert werden. Ein Zyklus enthält, neben den beiden IDs, das Attribut \texttt{period}, welches den Turnus beschreibt, in dem sich die Kurseinheiten des Kurses wiederholen sollen, sowie die Anzahl an Wiederholungen, nach denen der Zyklus beendet sein soll (\texttt{cycle\_end}).

\begin{verbatim}
CREATE TABLE cycles (
    id           SERIAL     PRIMARY KEY,
    course       REFERENCES courses(id)    ON DELETE CASCADE,
    period       PERIOD,
    cycle_end    INTEGER    NOT NULL
)
\end{verbatim}

\subsection{course\_instructors}
Die Tabelle \texttt{course\_instructors} enthält die Zuteilungen aller Kursleiter zu den jeweiligen Kursen. Dabei referenziert die Spalte \texttt{course\_instructor} die IDs aus der Tabelle \texttt{users} sowie die Spalte \texttt{course} die IDs aus der Tabelle \texttt{courses}.

\begin{verbatim}
CREATE TABLE course_instructors (
    course_instructor    REFERENCES users(id)   ON DELETE CASCADE,
    course               REFERENCES courses(id) ON DELETE CASCADE,
    PRIMARY KEY (course_instructor, course)
)
\end{verbatim}

\subsection{course\_participants}
Die Tabelle \texttt{course\_participants} enthält die Zuteilungen aller Kursteilnehmer zu den jeweiligen Kursen. Dabei referenziert die Spalte \texttt{participant} die IDs aus der Tabelle \texttt{users} sowie die Spalte \texttt{course} die IDs aus der Tabelle \texttt{courses}.

\begin{verbatim}
CREATE TABLE course_participants (
    participant    REFERENCES users(id)   ON DELETE CASCADE,
    course         REFERENCES courses(id) ON DELETE CASCADE,
    PRIMARY KEY (participant, course)
)
\end{verbatim}

\subsection{course\_unit\_participants}
Die Tabelle \texttt{course\_unit\_participants} enthält die Zuteilungen aller Teilnehmer einer Kurseinheit zu den jeweiligen Kurseinheiten. Dabei referenziert die Spalte \texttt{participant} die IDs aus der Tabelle \texttt{users} sowie die Spalte \texttt{course\_unit} die IDs aus der Tabelle \texttt{course\_units}.

\begin{verbatim}
CREATE TABLE course_participants (
    participant    REFERENCES users(id)        ON DELETE CASCADE,
    course_unit    REFERENCES course_units(id) ON DELETE CASCADE,
    PRIMARY KEY (participant, course_unit)
)
\end{verbatim}

\subsection{system\_attributes}
Die Tabelle \texttt{system\_attributes} enthält von den Systemadministratoren festgelegte Attribute, welche der Funktionalität der Webanwendung dienen.

\begin{verbatim}
CREATE TABLE system_attributes (
    lock                 CHAR(1)        PRIMARY KEY    									 DEFAULT 'X'    UNIQUE
                                        CHECK (lock = 'X'),
    activation_type      ACTIVATION     NOT NULL,
    withdrawal_hours     INTEGER        NOT NULL,
    application_hours    INTEGER        NOT NULL,
    verifiation_key      VARCHAR(50)    NOT NULL,
    storage_interval     INTEGER        NOT NULL
)
\end{verbatim}

\subsection{customization\_data}
Die Tabelle \texttt{customization\_data} enthält keys und values zu Daten, die für den Webauftritt der Applikation relevant sind. Diese können von den Administratoren editiert werden. Dazu gehört das Attribut \texttt{css}, welches den Namen der CSS-Datei für die Applikation enthält, sowie das Attribut \texttt{title}, welches die den Titel der Applikation enthält.

\begin{verbatim}
CREATE TABLE customization_data (
    lock     CHAR(1)        PRIMARY KEY    DEFAULT 'X'    UNIQUE
                            CHECK (lock = 'X'),
    css      VARCHAR(30)    NOT NULL,
    title    VARCHAR(30)    NOT NULL
)
\end{verbatim}

\subsection{monetary\_statistics}
Die Tabelle \texttt{monetary\_statistics} beschreibt die Geldstatistik des Systems. Eine Zeile dieser Tabelle enthält die Information des Geldbetrages, welches ein bestimmter Nutzer an einem bestimmten Datum innerhalb eines Kurses gezahlt hat. Dabei muss der Betrag größer als Null sein. Die Nutzer und Kurse werden in den Spalten \texttt{user} bzw. \texttt{course} referenziert.

\begin{verbatim}
CREATE TABLE monetary_statistics (
    user      REFERENCES users(id)         ON DELETE CASCADE,
    course    REFERENCES courses(id)       ON DELETE CASCADE,
    amount    DECIMAL(6, 2)    NOT NULL    CHECK (amount > 0),
    date      DATE             NOT NULL,
    PRIMARY KEY (user, course)
)
\end{verbatim}

\section{Constraints}
Das Sicherstellen der Integrität der Datenbanktabellen erfolgt durch die nachfolgenden constraints:

\begin{itemize}
\item Mit Ausnahme von \texttt{course\_instructors}, \texttt{course\_participants}, \texttt{system\_attributes}, \texttt{customization\_data} und \texttt{monetary\_statistics} dient die Spalte \texttt{id} jeder Tabelle als Primärschlüssel und wird per \texttt{auto\_increment} generiert. Dies geschieht PostgreSQL-spezifisch durch Angabe des Typs \texttt{SERIAL}.
\item Aufgrund der Wichtigkeit einiger Spalten bezüglich der Funktionalität des Systems dürfen diese nicht leer stehen. Zu diesem Zweck wird das Schlüsselwort \texttt{NOT NULL} verwendet.
\item Das Löschen eines Tupels aus den Tabellen \texttt{users}, \texttt{courses} oder \texttt{course\_units} führt zum kaskadierenden Löschen der Datensätze, die über foreign key constraints verbundenen sind. Durch Verwendung des SQL-Befehls ON DELETE CASCADE bei den Attributen \texttt{users(id)}, \texttt{courses(id)} und \texttt{course\_units(id)}, wirkt sich dies wie folgt aus:

\begin{itemize}
\item Wird ein Tupel aus der Tabelle \texttt{users} gelöscht, werden alle Tupel aus \texttt{courses}, \texttt{addresses}, \texttt{course\_instructors}, \texttt{course\_participants}, \texttt{course\_unit\_participants} und \texttt{monetary\_statistics} entfernt, bei denen das referenzierte Attribut der \texttt{id} des gelöschten Tupels aus der Tabelle \texttt{users} gleicht.

\item Wird ein Tupel aus der Tabelle \texttt{courses} gelöscht, werden alle Tupel aus \texttt{course\_units}, \texttt{cycles}, \texttt{course\_instructors}, \texttt{course\_participants} und \texttt{monetary\_statistics} entfernt, bei denen das referenzierte Attribut der \texttt{id} des gelöschten Tupels aus der Tabelle \texttt{courses} gleicht.

\item Wird ein Tupel aus der Tabelle \texttt{course\_units} gelöscht, werden alle Tupel aus \texttt{addresses} und \texttt{course\_unit\_participants} entfernt, bei denen das referenzierte Attribut der \texttt{id} des gelöschten Tupels aus der Tabelle \texttt{course\_units} gleicht.
\end{itemize}

\item Um eine Spalte initial mit einem default-Wert zu belegen wird das Schlüsselwort \texttt{DEFAULT} verwendet, gefolgt von dem entsprechenden Wert.

\item Die constraints \texttt{UNIQUE} und \texttt{CHECK} werden, aufgrund der individuellen Bedeutung für die jeweilige Tabelle, in deren Beschreibung genauer erklärt.

\item Die Tabellen \texttt{system\_attributes} und \texttt{customization\_data} enthalten jeweils das Attribut \texttt{lock}. Dadurch soll sichergestellt werden, dass diese beiden Tabellen, welche ausschließlich system- und konfigurations-spezifische Daten beinhalten, jeweils nur eine Zeile enthalten.
\end{itemize}

\section{Enums}
Folgende Attribute können, innerhalb des Systems, nur ausgewählte, vordefinierte Werte annehmen. Zu diesem Zweck wird, spezifisch für PostgreSQL, jeweils ein Enum erstellt.

\subsection{\texttt{users(gender)}}

\begin{verbatim}
CREATE TYPE gender AS ENUM (
    'male', 'female'
)
\end{verbatim}

\subsection{\texttt{users(role)}}

\begin{verbatim}
CREATE TYPE role AS ENUM (
    'registered_user', 'course_instructor', 'administrator', 'inactive'
)
\end{verbatim}

\subsection{\texttt{cycle(period)}}

\begin{verbatim}
CREATE TYPE period AS ENUM (
    'months', 'weeks', 'days', 'hours'
)
\end{verbatim}

\subsection{\texttt{system\_attributes(activation\_type)}}

\begin{verbatim}
CREATE TYPE activation AS ENUM (
    'e-mail', 'admin', 'complete'
)
\end{verbatim}
