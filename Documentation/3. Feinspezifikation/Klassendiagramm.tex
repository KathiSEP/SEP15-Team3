\chapter{Klassenbeschreibung und Klassendiagramm}
	\newcommand{\class}[1]{\paragraph{Klasse #1:}\ \\ }
	\newcommand{\interface}[1]{\paragraph{Interface #1:}\ \\ }
	\newcommand{\method}[1]{\textcolor{blue}{#1}}
	\newcommand{\kursiv}[1]{{\it #1}}
	\newcommand{\override}{{\it @Override}\ \\}
	
	Dieses Kapitel dient der Aufführung der Klassenbeschreibung und der der Darstellung des UML - Klassendiagramms der Anwendung \textbf{ofCourse}.
	Um eine bessere Übersicht und Strukturierung zu erhalten, wird das ganze Projekt in Packages aufgeteilt.\\
	Die Packagestruktur ist wie folgt aufgebaut: de.ofCourse.PACKAGENAME.\\
	Eine detaillierte Beschreibung der Klassen ist in dem beigefügten Dokument \kursiv{Klassenbeschreibung OfCourse} zu finden.\\
	Die verwendeten Synchronizations, Injections und Interfaces werden im Folgenden kurz zusammenfassend
	beschrieben.
	\begin{itemize}
		\item \textbf{Synchronizations}\\
		Die Anwendung \textbf{ofCourse} kommt, mit einer einzigen Ausnahme ohne den Modifier\kursiv{synchronized} aus. Diese Ausnahme stellt die Klasse DatabaseConnectionManager mit ihren Methoden \kursiv{getConnection()}  und \kursiv{releaseConnection()} dar.
		\item \textbf{Injections}\\
		Alle Klassen des Package \kursiv{action}, außer die Klasse \kursiv{SessionUser}, die Klasse
		\kursiv{Mail} und die Klasse \kursiv{Footer} enthalten ein \kursiv{private SessionUser sessionUser} Attribut, welches mit \kursiv{@ManagedProperty("\#sessionUser")} gekennzeichnet
		ist. Somit wird in diese Klassen die \kursiv{SessionUser} Klasse
		über Injection eingebunden.
		In der Klasse \kursiv{ContactUsers} wird zusätzlich zu der oben genannten Injection, noch die Klasse \kursiv{Mail}
		eingebunden.
		\item \textbf{Interfaces}
		\begin{itemize}
			\item \textbf{PaginationInteraction}
			\item \textbf{SessionInteraction}
			\item \textbf{Transaction}
		\end{itemize}
	\end{itemize}
	
	
	\section{Klassendiagramm}
	\begin{tiny}
		Diagramm: SeSc und TF\\
	\end{tiny}\\
	In der nachfolgenden Abbildung \ref{fig:classdiag} ist das UML-Klassendiagramm des ofCourse-Systems dargestellt.\\
	Um sich dieses Diagramm genauer ansehen zu können, muss es mit einem PDF-Reader
	geöffnet werden, der eine Zoom-Funktion besitzt.
	
	\begin{figure}[h]
		\centering
		\includegraphics[width=1\linewidth]{Grafiken/Klassendiagramm}
		\caption{UML-Klassendiagramm des ofCourse-Systems}
		\label{fig:classdiag}
	\end{figure}
	
